\thispagestyle{plain}

\section*{Kurzfassung}

Der inverse Faraday Effekt ist die Anregung einer Magnetisierung in einem Material durch die Bestrahlung mit
zirkular polarisiertem Licht. Als Mott-Isolator wird ein schwach leitendes Material bezeichnet, dessen
niedrige Leitfähigkeit aus einer nicht vernachlässigbaren Elektron-Elektron-Wechselwirkung resultiert.
Das Ziel dieser theoretischen Bachelorarbeit ist es, den Einfluss des inversen Faraday Effekts auf einen Mott-Isolator zu untersuchen.
Dazu werden zunächst theoretische Grundlagen der Quantenmechanik, verschiedene
Isolator-Modelle und eine Herleitung des inversen Faraday-Effekts in Metallen und Isolatoren vorgestellt.
Im nächsten Schritt wird ein Mott-Isolator durch ein Festkörpersystem, bestehend aus vier Gitterplätzen,
mit Hilfe des Hubbard-Modells modelliert und analysiert. Anschließend wird ein hochfrequent rotierendes E-Feld,
welches das für den inversen Faraday Effekt benötigte zirkular polarisierte Licht repräsentiert, im Isolator eingeschaltet.
Es wird gezeigt, dass kein Strom und somit keine Magnetisierung durch den inversen Faraday Effekt in dem
modellierten Mott-Isolator angeregt wird. Aus den Ergebnissen wird schlussgefolgert,
dass das Festkörpersystem aus vier Gitterplätzen ausreichend groß zur grundsätzlichen
Repräsentierung eines Mott-Isolators, aber nicht groß genug für die Untersuchung des inversen Faraday Effekts ist
und daher erweitert werden sollte.

\section*{Abstract}

\begin{english}
The inverse Faraday effect describes the excitation of magnetization in a material by irradiation with
circularly polarized light. Mott insulators are weakly conductive materials, whose
low conductivity results from a significant electron-electron-interaction.
The objective of this theoretical bachelor thesis is to examine the influence of the
inverse Faraday effect in a Mott insulator.
Therefore, theoretical foundations of quantum mechanics,
insulator models and a derivation of the inverse Faraday effect in metals and insulators are presented.
Moreover, a Mott insulator is modeled by a four-grid-solid-state system using the Hubbard model.
Subsequently, a rotating E-field, which represents the circularly polarized light required for the inverse Faraday effect, is added to the insulator.
It is pointed out that no current and thus no magnetization is induced by the inverse Faraday effect in the
modeled Mott insulator. The results indicate that the solid body system of four grating areas is sufficiently large to the fundamental
representation of a Mott insulator, however, not large enough to study the inverse Faraday effect in detail.
\end{english}
