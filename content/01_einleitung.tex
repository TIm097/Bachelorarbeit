\chapter{Einleitung}
\label{ch:einleitung}

Der inverse Faraday Effekt ist ein bedeutendes Phänomen in der Magnetooptik. Er beschreibt
die Anregung einer Magnetisierung in einem Material durch die Bestrahlung mit polarisiertem Licht
und kann somit Anwendungen in technischen Bereichen finden, wie z.B. in der Daten-Speicherung und -Übertragung.
Zudem kann der Effekt zur Erzeugung von Spinwellen genutzt werden, da diese entstehen,
wenn die Magnetisierung in einem magnetisch geordneten Material aus dem Gleichgewicht gebracht wird.
In einem Experiment mit dem Titel "Magnon Accumulation by Clocked Laser Excitation as Source of
Long-Range Spin Waves in Transparent Magnetic Films"\cite{jäckl} wurden beispielsweise ferri-magnetische
Eisen-Granat-Proben mit Laserpulsen angeregt, um kohärente Spinwellen zu generieren.
Dabei diente zirkular polarisiertes Laserlicht mittels des inversen Faraday Effekts als externe periodische Kraft auf
die magnetischen Momente der Proben.
Die Verwendung von ferri-magnetischen Eisen-Granaten ist auf ihre isolierende und somit transparente Eigenschaft
zurückzuführen.  \cite{hertel, jäckl}

Im Rahmen dieser theoretischen Bachelorarbeit wird der inverse Faraday Effekt in einem Mott-Isolator modelliert und analysiert, um
Erkenntnisse für die Erzeugung einer Magnetisierung durch polarisiertes Licht, beispielsweise in Bezug auf Spinwellen-Experimente, zu erhalten.
Dazu werden zunächst theoretische Grundlagen und Konventionen zum Verständnis der physikalischen und mathematischen Inhalte
der Arbeit in Kapitel \ref{ch:theorie} dargelegt. Im darauffolgenden Kapitel \ref{ch:isolatormodelle} werden verschiedene theoretische
Isolatoren und Isolator-Modelle vorgestellt, wobei der Schwerpunkt auf dem Mott-Isolator liegt.
Kapitel \ref{ch:ife} umfasst eine kurze Herleitung zur angeregten Magnetisierung durch den inversen Faraday Effekt in Metallen und Isolatoren.
Die Modellierungen und Ergebnisse dieser Arbeit werden in Kapitel \ref{ch:ergebnisse} präsentiert. Hierbei wird mehrfach die Eignung der Modellierung
des Mott-Isolators anhand der theoretischen Erwartungen aus den Kapiteln \ref{ch:theorie} und \ref{ch:isolatormodelle} überprüft.
Darüber hinaus wird die Abhängigkeit der angeregten Magnetisierung im Mott-Isolator von der Frequenz und der Amplitude des
polarisierten Lichts analysiert.
