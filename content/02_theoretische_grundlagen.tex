\chapter{Theoretische Grundlagen und Konventionen}

Vektoren fett
Im Folgenden wird in natürlichen Einheiten gerechnet, das heißt unter anderem
\begin{align}
  \hbar = 1.
\end{align}
ortsraum

\section{Die Schrödingergleichung}
\label{sec:schroedingergleichung}

Die zeitliche Entwicklung eines quantenmechanischen, nicht-relativistischen Zustandes $\ket{\Psi}$ ist durch die Schrödingergleichung
\begin{align}
  \symup{i} \frac{\partial}{\partial t} \ket{\Psi(\symbf{x},t)} = H(\symbf{x},t) \ket{\Psi(\symbf{x},t)}
  \label{eqn:schroedingergleichung}
\end{align}
gegeben. Dabei ist
\begin{align}
  H(\symbf{x},t) = -\frac1{2 m} \symbf{\nabla}^2 + V(\symbf{x},t)
  \label{eqn:hamiltonian}
\end{align}
der Hamiltonoperator mit einem beliebigen Potential $V(\symbf{x},t)$.
Wenn das Potential $V$ und somit auch der Hamiltonian $H$ zeitunabhängig ist, werden in der Schrödingergleichung \eqref{eqn:schroedingergleichung} Ort $\symbf{x}$ und Zeit $t$ mit dem Ansatz
\begin{align}
  \Psi(\symbf{x},t) = \Phi(\symbf{x})f(t)
  \label{eqn:separation}
\end{align}
separiert. Die daraus resultierende stationäre Schrödingergleichung
\begin{align}
  H \ket{\Phi(\symbf{x})} = E \ket{\Phi(\symbf{x})}
  \label{eqn:statschroedinger}
\end{align}
ist nur noch vom Ort $\symbf{x}$ abhängig und die Wellenfunktionen $\Psi(\symbf{x},t)$, die die Schrödingergleichung \eqref{eqn:schroedingergleichung} lösen, sind Eigenzustände des Hamiltonoperators $H$ zu den Eigenwerten $E$.
Wenn das Potential $V$ und der Hamiltonoperator $H$ zeitabhängig sind, kann die Schrödingergleichung \eqref{eqn:schroedingergleichung} nicht separiert werden, sondern in den meisten Fällen nur numerisch, zum Beispiel durch ein Integrationsverfahren, gelöst werden.
Nähere Informationen zur Herleitung und zur Lösung der Schrödingergleichung können \cite{schwabl} entnommen werden.

\section{Quantenmechanische Operatoren und die zweite Quantisierung}

In der Quantenmechanik sind den Observablen der klassischen Physik Operatoren zugeordnet.
Dieses Postulat wird als Korrespondenzprinzip bezeichnet.
\cite{schwabl}

Die zweite Quantisierung ist eine Darstellung von Operatoren, die sich vor allem für die Betrachtung von fermionischen Vielteilchensystemen eignet.
Grundlage für diese Darstellung sind die auf dem Fockraum definierten fermionischen Erzeugungs- und Vernichtungsoperatoren $c^\dag$ und $c$.
Diese ändern die Teilchenzahl eines fermionischen Vielteilchenzustands gemäß
\begin{align}
  c_i^{\phantom{\dag}} \ket{N_1, N_2, ..., N_i, ...} & = \pm n_i \ket{N_1, N_2, ..., N_i-1, ...} \label{eqn:vernichter}\\
  c_i^\dag \ket{N_1, N_2, ..., N_i, ...} & = \pm (1-N_i) \ket{N_1, N_2, ..., N_i+1, ...} \label{eqn:erzeuger}
\end{align}
um $1$. Dabei ist $N_i$ die Besetzungszahl des i-ten Zustands, die aufgrund des Pauliprinzips nur die Werte $0$ und $1$ annehmen kann.
Das Vorzeichen in \eqref{eqn:vernichter} und \eqref{eqn:erzeuger} entsteht durch die Antisymmetrie des Zustands. \cite{cy}

Im Folgenden werden Operatoren und Zustände ausschließlich in Ortsdarstellung angegeben und verwendet.

\section{Der Besetzungszahloperator}

Alle fermionischen Vielteilchenzustände sind Eigenfunktionen der Besetzungszahloperatoren
\begin{align}
  n_i = c_i^\dag c_i^{\phantom{\dag}},
  \label{eqn:besetzer}
\end{align}
wobei die zugehörigen Eigenwerte jeweils die Besetzungszahlen $n_i$ der i-ten Zustände sind.
Der Wertebereich der Eigenwerte ist daher ebenfalls auf $0$ und $1$ beschränkt.
\cite{cy}

\section{Der Spinoperator}

Die Indizes der Erzeugungs- und Vernichtungsoperatoren werden im Folgenden um einen zusätzlichen Parameter $\sigma$, der die Spinausrichtung eines Elektrons angibt, erweitert.
Die Ein-Teilchen-Zustände des fermionischen Vielteilchensystems spalten sich dadurch jeweils in zwei im Ort entartete Zustände auf, die sich nur in der Spin-Magnetquantenzahl $\sigma$ unterscheiden.
Der Parameter $\sigma$ kann die Werte $\sigma = \frac12$, für Spin Up ($\uparrow$), und $\sigma = -\frac12$, für Spin Down ($\downarrow$), annehmen.

Um den Gesamtspin $S$ eines fermionischen Vielteilchensystems zu ermitteln, wird zunächst der vektorartige Spinoperator für das Einteilchensystem
\begin{align}
  S & = \begin{pmatrix*}[r]
    S_x \\
    S_y \\
    S_z
  \end{pmatrix*}
  \label{eqn:spinoperator}
\end{align}
eingeführt. Die Quantisierungsachse wird im Folgenden in z-Richtung gelegt, sodass die Einteilchenzustände Eigenfunktionen des $S_z$-Operators zu den Eigenwerten $\sigma$ sind.
Der $S_x$- und der $S_y$-Operator werden dann mit
\begin{align}
  S_x & = \frac12 \left(S_+ + S_- \right) & S_y &= \frac1{2i} \left(S_+ - S_- \right)
  \label{eqn:leiteroperatorenausdruck}
\end{align}
durch die Spin-Leiteroperatoren $S_+$ und $S_-$ ausgedrückt. Die Wirkung der Spin-Leiteroperatoren auf einen fermionischen Einteilchenzustand mit Spin Up oder Spin Down ist durch
\begin{align}
  S_+ \ket{\uparrow} & = 0 & S_+ \ket{\downarrow} & = \ket{\uparrow} \label{eqn:spluswirkung}\\
  S_- \ket{\uparrow} & = \ket{\downarrow} & S_- \ket{\downarrow} & = 0 \label{eqn:sminuswirkung}
\end{align}
gegeben. Das Anwenden des $S_+$-Operators auf einen Einteilchenzustand entspicht dem Vernichten eines Elektrons mit Spin Down gefolgt vom Erzeugen
eines Elektrons mit Spin Up. Dies gilt auch für den $S_-$-Operator mit vertauschten Spins. Daraus folgt Darstellung der Spin-Leiteroperatoren in zweiter Quantisierung
\begin{align}
  S_+ & = c_\uparrow^\dag c_\downarrow^{\phantom{\dag}} & S_- & = c_\downarrow^\dag c_\uparrow^{\phantom{\dag}}. \label{eqn:leitersecondquant}
\end{align}
Die Eigenwertgleichungen für den $S_z$-Operator
\begin{align}
  S_z \ket{\uparrow} &= \frac12 \ket{\uparrow} & S_z \ket{\downarrow} &= -\frac12 \ket{\downarrow}
\end{align}
führen auf die Darstellung des Operators in zweiter Quantisierung
\begin{align}
  S_z = \frac12 (n_\uparrow -  n_\downarrow) = \frac12 (c_\uparrow^\dag c_\uparrow^{\phantom{\dag}} - c_\downarrow^\dag c_\downarrow^{\phantom{\dag}}) \label{eqn:szsecondquant}.
\end{align}
\cite{schwabl}

Aus den Gleichungen \eqref{eqn:spinoperator} und \eqref{eqn:leiteroperatorenausdruck} folgt der $S^2$-Operator für ein fermionisches Einteilchensystem
\begin{align}
  S^2 & = S_z^2 + \frac12 \left( S_+ S_- + S_- S_+ \right)
  \label{eqn:squadoperatorein}
\end{align}
Durch summieren der Einteilchenoperatoren $S_{+,i}$, $S_{-,i}$ und $S_{z,i}$ über alle Zustandsniveaus $i$ folgt aus den Gleichungen \eqref{eqn:squadoperatorein} die Darstellung des $S^2$-Operators in zweiter Quantisierung für ein fermionisches Vielteilchensystem
\begin{align}
  S^2 & = \frac14 \left(\sum_{i} n_{\uparrow,i} - n_{\downarrow,i}\right)^2 +
  \frac12 \left( \sum_{i,j} c_{\uparrow,i}^\dag c_{\downarrow,i}^{\phantom{\dag}} \cdot c_{\downarrow,j}^\dag c_{\uparrow,j}^{\phantom{\dag}} + c_{\downarrow,i}^\dag c_{\uparrow,i}^{\phantom{\dag}} \cdot c_{\uparrow,j}^\dag c_{\downarrow,j}^{\phantom{\dag}} \right).
  \label{eqn:squadoperatorviel}
\end{align}
Die Vielteilchenzustände sind Eigenfunktionen des $S^2$-Operators zu den Eigenwerten $S(S+1)$.
\cite{???}
