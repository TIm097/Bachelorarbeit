\chapter{Theoretische Grundlagen und Konventionen}

Dieses Kapitel beschäftigt sich mit einigen Grundlagen der Quantenmechanik, darunter die zentralen Postulate
und die Darstellungen von fermionischen Zuständen und Operatoren in erster und zweiter Quantisierung.
Es wird in atomaren Einheiten gerechnet, das heißt unter anderem
\begin{align}
  \hbar = \symup{e} = \symup{m_e} = 1.
\end{align}
Vektoren sind durch fettgedruckte Symbole, z.B. $\symbf{x}$ für den Ortsvektor, gekennzeichnet.

\section{Die Schrödingergleichung}
\label{sec:schroedingergleichung}

In der Quantenmechanik sind den Observablen der klassischen Physik Operatoren zugeordnet.
Die zeitliche Entwicklung eines quantenmechanischen, nicht-relativistischen Zustandes in Ortsdarstellung $\Psi(\symbf{x}_1,...,\symbf{x}_N,t)$ in einem physikalischen System aus $N$ Teilchen ist durch die Schrödingergleichung
\begin{align}
  \symup{i} \frac{\partial}{\partial t} \Psi(\symbf{x}_1,...,\symbf{x}_N,t) = H(\symbf{x}_1,...,\symbf{x}_N,t) \Psi(\symbf{x}_1,...,\symbf{x}_N,t)
  \label{eqn:schroedingergleichung}
\end{align}
gegeben. Dabei ist
\begin{align}
  H(\symbf{x}_1,...,\symbf{x}_N,t) = \sum_{n=1}^{N} \left( -\frac1{2 m_n} \symbf{\nabla}_n^2 \right) + V(\symbf{x}_1,...,\symbf{x}_N,t)
  \label{eqn:hamiltonian}
\end{align}
der Hamiltonoperator oder auch Hamiltonian in Ortsdarstellung und $V(\symbf{x}_1,...,\symbf{x}_N,t)$ ein Potential, welches das System charakterisiert.
Wenn das Potential $V$ und somit auch der Hamiltonian $H$ zeitunabhängig ist, werden in der Schrödingergleichung \eqref{eqn:schroedingergleichung} Ort $\symbf{x}$ und Zeit $t$ mit dem Ansatz
\begin{align}
  \Psi(\symbf{x}_1,...,\symbf{x}_N,t) = \Phi(\symbf{x}_1,...,\symbf{x}_N)f(t)
  \label{eqn:separation}
\end{align}
separiert. Die daraus resultierende stationäre Schrödingergleichung
\begin{align}
  H \Phi_\alpha(\symbf{x}_1,...,\symbf{x}_N) = E_\alpha \Phi_\alpha(\symbf{x}_1,...,\symbf{x}_N)
  \label{eqn:statschroedinger}
\end{align}
ist nur noch von den Orten $\symbf{x}_1,...,\symbf{x}_N$ der Teilchen abhängig und die Wellenfunktionen $\Psi_\alpha(\symbf{x}_1,...,\symbf{x}_N,t)$, welche die Schrödingergleichung \eqref{eqn:schroedingergleichung} lösen, sind Eigenzustände des Hamiltonoperators $H$ zu den Eigenwerten $E_\alpha$.
Der zeitabhängige Teil der Wellenfunktionen $\Psi_\alpha(\symbf{x}_1,...,\symbf{x}_N,t)$ ist gegeben durch
\begin{align}
  f_\alpha(t) = \symup{e}^{-\symup{i} E_\alpha t}.
\end{align}
Wenn das Potential $V$ und der Hamiltonoperator $H$ zeitabhängig sind, kann die Schrödingergleichung \eqref{eqn:schroedingergleichung} nicht separiert werden, sondern in den meisten Fällen nur numerisch, zum Beispiel durch ein Integrationsverfahren, gelöst werden.
Die Aufenthaltswahrscheinlichkeit eines Teilchen $n$ im Volumenelement $\symup{d}^3x_n$ zum Zeitpunkt $t$ ist gegeben durch
\begin{align}
  P_n = \lvert \Psi(\symbf{x}_1,...,\symbf{x}_N,t) \rvert^2 \symup{d}^3x_n.
\end{align}
Nähere Informationen zur Herleitung und Lösung der Schrödingergleichung und zu den Postulaten der Quantenmechanik können Quelle \cite{schwabl} entnommen werden.

\section{Fermionische Niveau-Systeme und die erste und zweite Quantisierung}

Ein fermionisches Niveau-System besteht aus einer Menge an Besetzungsniveaus, denen jeweils eine Menge $k$ an Quantenzahlen zugeordnet sind, und einer Anzahl $N_e$ an Fermionen, die auf den Niveaus verteilt sind.
Auf jedem Niveau kann sich wegen des Pauliprinzips(footnote) maximal ein Fermion befinden.

Sei der Zustand $\ket{k_i}^i$ eine Lösung der Schrödingergleichung für ein Teilchen in einem gegebenen, fermionischen Niveau-System.
Dann wird ein Vielteilchenzustand, bestehend aus $N_e$ Teilchen, in diesem Niveau-System gemäß
\begin{align}
  \ket{\Psi} = \frac{1}{\sqrt{N_e}} \sum_{P} (-1)^{\chi_P} \bigotimes\limits_{i=1}^{N_e} \ket{k_{P(i)}}^i
  \label{eqn:vtdarstellung1quant}
\end{align}
über das Tensorprodukt(footnote) der Einteilchen-Basiszustände $\ket{k_i}^i$ in erster Quantisierung dargestellt.
Dabei wird über alle Permutationen $P$ der Basiszustände summiert und $\chi_P$ gibt jeweils die Anzahl an Transmissionen, die zur Permutation $P$ führen, an.
Die Darstellung der Vielteilchenzustände in erster Quantisierung aus Gleichung \eqref{eqn:vtdarstellungeinteilchenbasis} wird im Folgenden mit
\begin{align}
  \ket{\Psi} = \ket{k_1,...,k_{N_e}}
  \label{eqn:vtdarstellung1quantkurz}
\end{align}
abgekürzt.

In der zweiten Quantisierung werden die Vielteilchenzustände nicht durch ein Tensorprodukt der Einteichenzustände, sondern über die Besetzungszahlen der Niveaus gemäß
\begin{align}
  \ket{\Psi} = \ket{N_1,N_2,...,N_i,...}
\end{align}
dargestellt, wobei $N_i \in \{ 0,1 \}$ die Besetzungszahl des i-ten Niveaus ist. Diese Darstellung folgt aus der Ununterscheidbarkeit der Teilchen.
Um einen fermionischen Operator in zweiter Quantisierung darzustellen, werden zunächst die auf dem Fockraum(footnote) definierten fermionischen Erzeugungs- und Vernichtungsoperatoren $c^\dag$ und $c$ eingeführt.
Diese ändern die Teilchenzahl eines fermionischen Vielteilchenzustands mit den Vorschriften
\begin{align}
  c_i^{\phantom{\dag}} \ket{N_1,N_2,...,N_i,...} & = \pm N_i \ket{N_1,N_2,...,N_i-1,...} \label{eqn:vernichter}\\
  c_i^\dag \ket{N_1,N_2,..., N_i,...} & = \pm (1-N_i) \ket{N_1,N_2,...,N_i+1,...} \label{eqn:erzeuger}
\end{align}
um $1$. Das Vorzeichen in \eqref{eqn:vernichter} und \eqref{eqn:erzeuger} resultiert aus der Antisymmetrie der fermionischen Wellenfunktion und hängt von der Anzahl und Besetzung der Niveaus im Vielteilchenzustand ab.
Ein beliebiger Einteilchen-Operator eines fermionischen Vielteilchensystems ist in erster Quantisierung durch
\begin{align}
  A = \sum_{i=1}^{N_e} A(\symbf{r}_i)
\end{align}
gegeben, wobei über alle Teilchen im System summiert wird und $A(\symbf{r}_i)$ auf das i-te Teilchen des Vielteilchensystems wirkt.
Der Operator ergibt sich in zweiter Quantisierung durch eine Linearkombination der Erzeugungs- und Vernichtungsoperatoren
\begin{align}
  A = \sum_{m,n} A_{mn} c_m^\dag c_n^{\phantom{\dag}},
\end{align}
wobei die Koeffizienten
\begin{align}
  A_{mn} = \bra{k_m} A \ket{k_n}
  \label{eqn:matrixelemente}
\end{align}
die Matrixelemente des Operators bezogen auf die Einteilchen-Basis-Zustände $\ket{k_m}$ bzw. $\ket{k_n}$ sind.
Weitere Informationen zu Vielteilchensystemen und zur zweiten Quantisierung können Quelle \cite{cy} entnommen werden.

\section{Der Besetzungszahloperator}

Die fermionischen Vielteilchenzustände sind Eigenfunktionen der Besetzungszahloperatoren
\begin{align}
  n_i = c_i^\dag c_i^{\phantom{\dag}},
  \label{eqn:besetzer}
\end{align}
wobei die zugehörigen Eigenwerte jeweils die Besetzungszahlen $N_i$ der Niveaus sind.
Der Wertebereich der Eigenwerte ist aufgrund des Pauliprinzips auf $N_i \in \{ 0,1 \}$ beschränkt.

\section{Der \texorpdfstring{$\symbf{S}^2$}{TEXT}- Operator}

Die Indizes der Erzeugungs- und Vernichtungsoperatoren werden im Folgenden um einen zusätzlichen Parameter $\sigma$, der die Spinausrichtung eines Fermions angibt, erweitert.
Die Einteilchen-Zustände eines fermionischen Vielteilchensystems spalten sich dadurch jeweils in zwei im Ort entartete Zustände auf, die sich nur in der Spinausrichtung $\sigma$ unterscheiden.
Der Parameter $\sigma$ kann die Werte $\sigma = \frac12$, für Spin Up ($\uparrow$), und $\sigma = -\frac12$, für Spin Down ($\downarrow$), annehmen.
\cite{schwabl}

Die Quantisierungsachse wird in z-Richtung gelegt, sodass der $S_x$- und der $S_y$-Operator durch die
Leiteroperatoren $S_+$ und $S_-$ ersetzt werden können. Der $S^2$-Operator eines fermionischen Einteilchen-Systems lautet in erster Quantisierung
\begin{align}
  S^2 & = S_z^2 + \frac12 \left( S_+ S_- + S_- S_+ \right).
  \label{eqn:squadoperatorein}
\end{align}
Die Wirkung des $S_z$-Operators und der Leiteroperatoren $S_+$ und $S_-$ ist durch
\begin{align}
  S_z \ket{\uparrow} & = 0.5\ket{\uparrow} & S_z \ket{\downarrow} & = -0.5\ket{\downarrow} \label{eqn:szwirkung}\\
  S_+ \ket{\uparrow} & = 0 & S_+ \ket{\downarrow} & = \ket{\uparrow} \label{eqn:spluswirkung}\\
  S_- \ket{\uparrow} & = \ket{\downarrow} & S_- \ket{\downarrow} & = 0 \label{eqn:sminuswirkung}
\end{align}
gegeben. Das Anwenden des $S_+$-Operators auf einen Einteilchenzustand entspicht dem Vernichten eines Elektrons mit Spin Down gefolgt vom Erzeugen
eines Elektrons mit Spin Up. Dies gilt auch für den $S_-$-Operator mit vertauschten Spins. Daraus folgt Darstellung der Spin-Leiteroperatoren in zweiter Quantisierung
\begin{align}
  S_+ & = c_\uparrow^\dag c_\downarrow^{\phantom{\dag}} & S_- & = c_\downarrow^\dag c_\uparrow^{\phantom{\dag}}. \label{eqn:leitersecondquant}
\end{align}
Aus der Eigenwertgleichung \eqref{eqn:szwirkung} für den $S_z$-Operator folgt die Darstellung des Operators in zweiter Quantisierung
\begin{align}
  S_z = \frac12 (n_\uparrow -  n_\downarrow) = \frac12 (c_\uparrow^\dag c_\uparrow^{\phantom{\dag}} - c_\downarrow^\dag c_\downarrow^{\phantom{\dag}}) \label{eqn:szsecondquant}.
\end{align}
\cite{schwabl}

Durch summieren der Einteilchenoperatoren $S_{+,i}$, $S_{-,i}$ und $S_{z,i}$ über alle Niveaus $i$ bzw. $j$ folgt aus den Gleichungen \eqref{eqn:squadoperatorein}, \eqref{eqn:leitersecondquant}
und \eqref{eqn:szsecondquant} die Darstellung des $S^2$-Operators in zweiter Quantisierung für ein fermionisches Vielteilchensystem
\begin{align}
  S^2 & = \frac14 \left(\sum_{i} \left( n_{\uparrow,i} - n_{\downarrow,i}\right)\right)^2 +
  \frac12  \sum_{i,j}\left( c_{\uparrow,i}^\dag c_{\downarrow,i}^{\phantom{\dag}} \cdot c_{\downarrow,j}^\dag c_{\uparrow,j}^{\phantom{\dag}} +
  c_{\downarrow,i}^\dag c_{\uparrow,i}^{\phantom{\dag}} \cdot c_{\uparrow,j}^\dag c_{\downarrow,j}^{\phantom{\dag}} \right) .
  \label{eqn:squadoperatorviel}
\end{align}
Die Vielteilchenzustände sind Eigenfunktionen des $S^2$-Operators zu den Eigenwerten $S(S+1)$.
\cite{???}

\section{Der Ladungsdichteoperator}

Für ein Vielteilchensystem aus Elektronen lässt sich ein Ladungsdichteoperator definieren, der sich gemäß
\begin{align}
  \rho = -\sum_{i,\sigma} n_{i,\sigma}
  \label{eqn:ladungsdichteoperator}
\end{align}
aus den Besetzungszahloperatoren darstellen lässt.
\cite{cy}


Footnotes und evtl Lehrbuchverweise

Kommutatoren c c und S+ -

Erwartungswertpostulat
