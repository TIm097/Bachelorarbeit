\chapter{Theoretische Grundlagen und Konventionen}
\label{ch:theorie}

In diesem Kapitel werden einige Grundlagen der Quantenmechanik, darunter die zentralen Postulate
und die Darstellungen von fermionischen Zuständen und Operatoren in erster und zweiter Quantisierung, vorgestellt.
Im Rahmen dieser Arbeit wird in atomaren Einheiten gerechnet, das heißt
\begin{align*}
  \hbar = \symup{e} = \symup{m_e} = \frac{1}{4\symup{\pi}\epsilon_0} = 1.
\end{align*}

\section{Die Postulate der Quantenmechanik}
\label{sec:schroedingergleichung}

In der Quantenmechanik werden die Messgrößen der klassischen Physik durch lineare, hermitesche Operatoren dargestellt.
Eine Observable in einem quantenmechanischen System ist durch den Erwartungswert des zugehörigen Operators $A$
\begin{align}
  \langle A \rangle (t) = \bra{\Psi(t)} A \ket{\Psi(t)}
\end{align}
bezogen auf den Zustand $\Psi$ charakterisiert.
Die zeitliche Entwicklung eines quantenmechanischen Zustands in Ortsdarstellung $\Psi(\vec{x}_1,...,\vec{x}_N,t)$ für ein System aus $N$ Teilchen ist durch die Schrödingergleichung
\begin{align}
  \symup{i} \frac{\partial}{\partial t} \Psi(\vec{x}_1,...,\vec{x}_N,t) = H(\vec{x}_1,...,\vec{x}_N,t) \Psi(\vec{x}_1,...,\vec{x}_N,t)
  \label{eqn:schroedingergleichung}
\end{align}
gegeben, wobei $H(\vec{x}_1,...,\vec{x}_N,t)$ der Hamiltonoperator in Ortsdarstellung ist.
Wenn dieser zeitunabhängig ist, werden in der Schrödingergleichung \eqref{eqn:schroedingergleichung} Ort $\vec{x}$ und Zeit $t$ mit dem Ansatz
\begin{align*}
  \Psi(\vec{x}_1,...,\vec{x}_N,t) = \varphi(\vec{x}_1,...,\vec{x}_N)f(t)
\end{align*}
separiert. Die daraus resultierende stationäre Schrödingergleichung
\begin{align}
  H \varphi_\alpha(\vec{x}_1,...,\vec{x}_N) = E_\alpha \varphi_\alpha(\vec{x}_1,...,\vec{x}_N)
  \label{eqn:statschroedinger}
\end{align}
ist nur noch von den Orten $\vec{x}_1,...,\vec{x}_N$ der Teilchen abhängig und die Wellenfunktionen $\Psi_\alpha(\vec{x}_1,...,\vec{x}_N,t)$, welche die Schrödingergleichung \eqref{eqn:schroedingergleichung} lösen, sind Eigenzustände des Hamiltonoperators $H$ zu den Eigenwerten $E_\alpha$.
Der ebenfalls resultierende zeitabhängige Teil der Schrödingergleichung ist gegeben durch
\begin{align}
  f_\alpha(t) = \symup{e}^{-\symup{i} E_\alpha t}.
\end{align}
In dem Fall, dass der Hamiltonoperator $H$ zeitabhängig ist, kann die Schrödingergleichung \eqref{eqn:schroedingergleichung} nicht separiert werden, sondern in den meisten Fällen numerisch, z.B. durch ein Integrationsverfahren, gelöst werden.
%Die Aufenthaltswahrscheinlichkeit eines Teilchen $n$ im Volumenelement $\symup{d}^3x_n$ zum Zeitpunkt $t$ ist gegeben durch
%\begin{align}
%  P_n = \lvert \Psi(\vec{x}_1,...,\vec{x}_N,t) \rvert^2 \symup{d}^3x_n.
%\end{align}
Diese und weitere Informationen zu den Postulaten der Quantenmechanik und zur Herleitung der Schrödingergleichung können dem Lehrbuch \cite{schwabl} entnommen werden.

\section{Fermionische Niveau-Systeme und die erste und zweite Quantisierung}

Ein fermionisches Niveau-System setzt sich aus einer Menge an Niveaus, denen jeweils ein vollständiger Satz $k$ an Quantenzahlen zugeordnet ist, und einer Anzahl $N_e$ an Fermionen, die auf den Niveaus verteilt sind, zusammen.
Auf jedem Niveau kann sich aufgrund des Pauliprinzips\footnote{Zwei Fermionen eines Vielteilchensystems unterscheiden sich in mindestens einer Quantenzahl.} maximal ein Fermion befinden.

Zu einem Einteilchen-Basiszustand $\ket{k_i}$, welcher die Schrödingergleichung für ein gegebenes fermionisches Niveau-System löst,
wird ein Vielteilchenzustand mit Hilfe des Tensorprodukts gemäß
\begin{align}
  \ket{\Psi} = \frac{1}{\sqrt{N_e}} \sum_{P} (-1)^{\chi_P} \bigotimes\limits_{i=1}^{N_e} \ket{k_{P(i)}}
  \label{eqn:vtdarstellung1quant}
\end{align}
in erster Quantisierung dargestellt. Die Summe läuft dabei über alle verschiedenen Permutationen $P$ der Basiszustände und $\chi_P$ gibt jeweils die Anzahl an Transmissionen, die zur jeweiligen Permutation $P$ führen, an.

In der zweiten Quantisierung werden die Vielteilchenzustände nicht durch ein Tensorprodukt der Einteichen-Basiszustände, sondern über die Besetzungszahlen der Niveaus mit
\begin{align}
  \ket{\Psi} = \ket{N_1,N_2,...,N_i,...}
\end{align}
dargestellt, wobei $N_i \in \{ 0,1 \}$ die Besetzungszahl des i-ten Niveaus ist. Diese Darstellung folgt aus der Ununterscheidbarkeit der Teilchen.
Um einen fermionischen Operator in zweiter Quantisierung auszudrücken, werden zunächst die fermionischen Erzeugungs- und Vernichtungsoperatoren $c^\dag$ und $c$ eingeführt.
Diese ändern die Teilchenzahl eines fermionischen Vielteilchenzustands mit den Vorschriften
\begin{align}
  c_i^{\phantom{\dag}} \ket{N_1,N_2,...,N_i,...} & = \pm N_i \ket{N_1,N_2,...,N_i-1,...} \label{eqn:vernichter}\\
  c_i^\dag \ket{N_1,N_2,..., N_i,...} & = \pm (1-N_i) \ket{N_1,N_2,...,N_i+1,...} \label{eqn:erzeuger}
\end{align}
um $1$ und genügen den Antikommutatorrelationen
\begin{align}
  \{ c_i^\dag, c_j^{\phantom{\dag}} \} & = \delta_{ij} & \{ c_i^{\phantom{\dag}}, c_j^{\phantom{\dag}} \} = \{ c_i^\dag, c_j^\dag \} = 0.
  \label{eqn:ckommutatoren}
\end{align}
Das Vorzeichen in den Vorschriften \eqref{eqn:vernichter} und \eqref{eqn:erzeuger} resultiert aus der Antisymmetrie der fermionischen Wellenfunktion und hängt von der Anzahl und Besetzung der Niveaus ab.
Ein beliebiger Einteilchenoperator eines fermionischen Vielteilchensystems ist in erster Quantisierung durch
\begin{align}
  A = \sum_{i=1}^{N_e} A(\vec{r}_i)
\end{align}
gegeben, wobei $A(\vec{r}_i)$ jeweils auf das i-te Teilchen eines Vielteilchenzustands wirkt.
In zweiter Quantisierung ergibt sich der Operator aus einer Linearkombination der Erzeugungs- und Vernichtungsoperatoren
\begin{align}
  A = \sum_{m,n} A_{mn} c_m^\dag c_n^{\phantom{\dag}},
\end{align}
wobei die Koeffizienten
\begin{align}
  A_{mn} = \bra{k_m} A \ket{k_n}
  \label{eqn:matrixelemente}
\end{align}
die Matrixelemente des Operators bezogen auf die Einteilchen-Basiszustände $\ket{k_m}$ bzw. $\ket{k_n}$ sind.
In den folgenden Abschnitten werden mehrere Operatoren vorgestellt und in zweiter Quantisierung ausgedrückt.
Dabei werden ausschließlich Vielteilchensysteme, bestehend aus Elektronen, betrachtet. \cite{czycholl}

\section{Der \texorpdfstring{$\symbf{S}^2$}{TEXT}- Operator}

Die Indizes der Erzeugungs- und Vernichtungsoperatoren werden im Folgenden um den zusätzlichen Parameter $\sigma$, der die Spinausrichtung eines Elektrons angibt, erweitert.
Die Einteilchen-Zustände eines Vielteilchensystems spalten sich dadurch jeweils in zwei im Ort entartete Zustände auf, die sich lediglich in der Spinausrichtung $\sigma$ unterscheiden.
Der Parameter $\sigma$ kann dabei die Werte $\sigma = \frac12$, für Spin-Up ($\uparrow$), und $\sigma = -\frac12$, für Spin-Down ($\downarrow$), annehmen.

Die Quantisierungsachse wird in z-Richtung gelegt, sodass der $S_x$- und der $S_y$-Operator durch die
Leiteroperatoren $S_+$ und $S_-$, welche die Kommutatorrelation
\begin{align}
  \left[ S_+ , S_- \right] & = 2 S_z
  \label{eqn:skommutator}
\end{align}
erfüllen, ersetzt werden können. Für ein Einteilchen-System lautet der $S^2$-Operator in erster Quantisierung
\begin{align}
  \symbf{S}^2 & = S_z^2 + \frac12 \left( S_+ S_- + S_- S_+ \right).
  \label{eqn:squadoperatorein}
\end{align}
%Die Wirkung des $S_z$-Operators und der Leiteroperatoren $S_+$ und $S_-$ ist durch
%\begin{align}
%  S_z \ket{\uparrow} & = \tfrac12\ket{\uparrow} & S_z \ket{\downarrow} & = -\tfrac12\ket{\downarrow} \label{eqn:szwirkung}\\
%  S_+ \ket{\uparrow} & = 0 & S_+ \ket{\downarrow} & = \ket{\uparrow}\phantom{\tfrac12} \label{eqn:spluswirkung}\\
%  S_- \ket{\uparrow} & = \ket{\downarrow} & S_- \ket{\downarrow} & = 0\phantom{\tfrac12} \label{eqn:sminuswirkung}
%\end{align}
%gegeben.
Das Anwenden des $S_+$-Operators auf einen Spin-Down-Einteilchenzustand entspicht dem Vernichten des Elektrons mit Spin-Down gefolgt vom Erzeugen
eines Elektrons mit Spin-Up. Dies gilt auch für den $S_-$-Operator mit vertauschten Spins. Daraus folgt die Darstellung der Spin-Leiteroperatoren in zweiter Quantisierung
\begin{align}
  S_+ & = c_\uparrow^\dag c_\downarrow^{\phantom{\dag}} & S_- & = c_\downarrow^\dag c_\uparrow^{\phantom{\dag}}. \label{eqn:leitersecondquant}
\end{align}
Der $S_z$-Operator liefert bei Anwendung auf einen Einteilchenzustand den Eigenwert $\sigma$, sodass er in zweiter Quantisierung gemäß
\begin{align}
  S_z = \frac12 (n_\uparrow - n_\downarrow) = \frac12 (c_\uparrow^\dag c_\uparrow^{\phantom{\dag}} - c_\downarrow^\dag c_\downarrow^{\phantom{\dag}}) \label{eqn:szsecondquant}.
\end{align}
dargestellt wird, wobei $n_\sigma = c_\sigma^{\dag}c_\sigma^{\phantom{\dag}}$ der Besetzungszahloperator ist.

Durch Summieren der Einteilchenoperatoren $S_{+}$, $S_{-}$ und $S_{z}$ über alle Niveaus $i$ bzw. $j$ folgt aus den Gleichungen \eqref{eqn:squadoperatorein}, \eqref{eqn:leitersecondquant}
und \eqref{eqn:szsecondquant} die Darstellung des $S^2$-Operators in zweiter Quantisierung für ein Vielteilchensystem aus Elektronen
\begin{align}
  \symbf{S}^2 & = \frac14 \left(\sum_{i} \left( n_{\uparrow,i} - n_{\downarrow,i}\right)\right)^2 +
  \frac12  \sum_{i,j}\left( c_{\uparrow,i}^\dag c_{\downarrow,i}^{\phantom{\dag}} \cdot c_{\downarrow,j}^\dag c_{\uparrow,j}^{\phantom{\dag}} +
  c_{\downarrow,i}^\dag c_{\uparrow,i}^{\phantom{\dag}} \cdot c_{\uparrow,j}^\dag c_{\downarrow,j}^{\phantom{\dag}} \right) .
  \label{eqn:squadoperatorviel}
\end{align}
Die Ein- und Vielteilchenzustände sind Eigenfunktionen des $S^2$-Operators zu den Eigenwerten $S(S+1)$.
\cite{schwabl,czycholl}

\section{Der Ladungsdichteoperator und der Stromoperator}
\label{sec:stromoperator}

Für ein Vielteilchensystem aus Elektronen, die sich an diskreten Gitterplätzen eines eindimensionalen Gitters befinden, wird der Ladungsdichteoperator
\begin{align}
  \rho_k = - \sum_{\sigma} c_{k,\sigma}^{\dag} c_{k,\sigma}^{\phantom{\dag}}
  \label{eqn:ladungsdichteoperator}
\end{align}
für den $k$-ten Gitterplatz in zweiter Quantisierung definiert.
Aus der Kontinuitätsgleichung für die eindimensionale elektrische Stromdichte $j$
\begin{align}
  \frac{\partial \rho}{\partial t} + \frac{\partial j}{\partial x} = 0
\end{align}
und der Heisenbergschen Bewegungsgleichung
\begin{align}
  \frac{\symup{d}\rho}{\symup{d}t} = \symup{i} \left[ H,\rho \right]
\end{align}
kann für ein eindimensionales System, welches durch den Hamiltonoperator $H$ beschrieben ist,
der Stromoperator hergeleitet werden. \cite{schwabl,czycholl}
