\chapter{Festkörper-Modelle}

evtl kurz Bändermodell

\section{Das Tight-binding-Modell und der Stromoperator}

Das allgemeine Bändermodell beruht auf der Annahme, dass sich die Elektronen in einem Festkörper unter dem Einfluss eines periodischen Gitterpotentials quasifrei bewegen.
Durch das Tight-Binding-Modell wird ein Festkörpersystem beschrieben, in welchem die Elektronen stark gebunden und lokalisiert sind und abgesehen von der Pauliabstoßung nicht miteinander in Wechselwirkung treten.
Für die Elektronen sind dann nur noch diskrete Sprünge zwischen den Gitterplätzen des Festkörpers möglich. Der Hamiltonoperator des Tight-Binding-Modells in zweiter Quantisierung lautet
\begin{align}
  H_\text{tb} = -J \sum_{i,\sigma} (c_{i+1,\sigma}^{\dag}c_{i,\sigma}^{\phantom{\dag}} + c_{i,\sigma}^{\dag}c_{i+1,\sigma}^{\phantom{\dag}})
  \label{eqn:hamiltontb}
\end{align}
für ein Festkörpersystem mit einem Elektronenband. Summiert wird dabei über alle Gitterpätze $i$ und Spinausrichtungen $\sigma$.
Der Term $c_{i+1,\sigma}^{\dag}c_{i,\sigma}^{\phantom{\dag}}$ lässt ein Elektron vom Gitterplatz $i$ zu $i+1$ springen und der Term $c_{i,\sigma}^{\dag}c_{i+1,\sigma}^{\phantom{\dag}}$ umgekehrt.
Der Vorfaktor $J$ ist eine energetische Konstante, welche die Stärke der Tight-Binding-Wechselwirkung im Hamiltonian beschreibt.
Stromoperator!!!!!!!!!!!!

\section{Das Hubbard-Modell}

Die bisher erwähnten Modelle vernachlässigen die Coulomb-Wechselwirkung der Elektronen untereinander.
Das Hubbard-Modell ist eine spezielle Erweiterung des Tight-Binding-modells, welche diese Wechselwirkung berücksichtigt.
Dies ist im Hamiltonoperator durch einen zusätzlichen Term, der eine starke Coulombabstoßung zwischen den Elektronen beschreibt, realisiert.
Im Folgenden wird die Abstoßung in erster Ordnung, also für Elektronen, die sich auf demselben Gitterpatz befinden, betrachtet.
Diese Näherung lässt sich dadurch begründen, dass das Coulombpotential mit $\frac1{r}$ abfällt. Der Hamiltonoperator des Hubbardmodells ist dann in zweiter Quantisierung durch
\begin{align}
  H_\text{hubb} = H_\text{tb} + U\sum_{i} \prod_{\sigma} n_{i,\sigma}.
  \label{eqn:hamiltonhubb}
\end{align}
gegeben, wobei erneut über die Gitterplätze $i$ und die Spinausrichtungen $\sigma$ summiert wird.

\section{Der Übergang zum Heisenberg-Austausch-Modell im Grenzfall des halbgefüllten Hubbard-Modells}

Die Besetzungswahrscheinlichkeit der Zustände im halbgefüllten Hubbardmodell, in denen sich jeweils ein Elektron auf jedem Gitterplatz befindet, steigt mit dem Wechselwirkungs-Potential $U$ an, da
die Doppelbesetzung eines Atoms mit dem Ansteigen des Potentials $U$ energetisch ungünstiger wird. Im Grenzfall $U \to \infty$ besteht der Hilbertraum des Festkörpersystems nur noch aus diesen Zuständen
und die Elektronen können sich nicht mehr zwischen den Gitterplätzen bewegen. Die Dynamik des Systems ist dann nur noch durch virtuelle Prozesse gegeben, in welchen die Spinausrichtung der Elektronen
benachbarter Gitterplätze getauscht oder beibehalten wird. Für einen virtuellen Prozesse zwischen zwei benachbarten Elektronen mit antiparalleler Spinausrichtung, z.B.
\begin{align}
  (\uparrow; \downarrow) &\to (\uparrow \downarrow ; 0) \to (\uparrow; \downarrow),
\end{align}
ergibt sich in zweiter Ordnung Störungstheorie die Energiedifferenz
\begin{align}
  \Delta E_{\uparrow \downarrow} = - \frac{2J^2}{U}.
  \label{eqn:Ediffantipar}
\end{align}
Ein Sprung zwischen zwei benachbarten Elektronen mit paralleler Spinausrichtung ist wegen des Pauliprinzips nicht möglich.
Der Energieunterschied beträgt daher
\begin{align}
  \Delta E_{\uparrow \uparrow} = 0.
  \label{eqn:Ediffpar}
\end{align}
Da die virtuellen Prozesse nur noch von der Spinausrichtung abhängen, wird das System mit dem Heisenberg-Austausch-Modell verglichen.
Der Hamiltonoperator dieses Modells ist durch
\begin{align}
  H_\text{spin} = t \sum_{i,j} (S_i S_j + C)
  \label{eqn:hamiltonspincj}
\end{align}
gegeben, wobei in beiden Summen über alle Gitterplätze $i$ und $j$ summiert wird. Durch das Vergleichen der Energiedifferenzen der unterschiedlichen Spinkonfigurationen in diesem Modell
\begin{align}
  \Delta E_{\uparrow \downarrow} = \bra{\uparrow \downarrow} H_\text{spin} \ket{\uparrow \downarrow} = t\left(C - \frac14\right) \\
  \Delta E_{\uparrow \uparrow} = \bra{\uparrow \uparrow} H_\text{spin} \ket{\uparrow \uparrow} = t\left(C + \frac14\right)
\end{align}
mit den Energiedifferenzen \eqref{eqn:Ediffantipar} und \eqref{eqn:Ediffpar} ergeben sich die Konstanten
\begin{align}
  t = \frac{4 J^2}{U}
\end{align}
und
\begin{align}
  C = -\frac14
\end{align}
für den Hamiltonian \eqref{eqn:hamiltonspincj}.
Im Grenzfall $U \to \infty$ geht der Hamilton-Operator des halbgefüllten Hubbard-Modells \eqref{eqn:hamiltonhubb} also in den Hamilton-Operator des Heisenberg-Austausch-Modells
\begin{align}
  H_\text{spin} = \frac{4 J^2}{U} \sum_{i,j} (S_i S_j - \frac14)
  \label{eqn:hamiltonspin}
\end{align}
über.

Eine mögliche Struktur der Arbeit sieht wie folgt aus:

\begin{enumerate}
    \item \textbf{Einleitung}\\
        In der \emph{kurzen} Einleitung wird die Motivation für die Arbeit
        dargestellt und ein Einblick in die kommenden Kapitel gegeben.
    \item \textbf{Theoretische Grundlagen}\\
        Alles was an theoretischen Grundlagen benötigt wird, sollte auch eher kurz gehalten werden.
        Statt Grundlagenwissen zu präsentieren, eher auf die entsprechenden Lehrbücher verweisen.
        Etwa: Tiefer gehende Informationen zur klassischen Mechanik entnehmen Sie bitte \cite{kuypers}.
    \item \textbf{Ergebnisse} \\
        Der eigentliche Teil der Arbeit, das was getan wurde.
    \item \textbf{Zusammenfassung und Ausblick} \\
        Zusammenfassung der Ergebnisse, Optimierungsmöglichkeiten, mögliche weitergehende Untersuchungen.
\end{enumerate}

Die Gliederung sollte auf der einen Seite nicht zu fein sein, auf der anderen Seite
sollten sich klar unterscheidende Abschnitte auch kenntlich gemacht werden.

In der hier verwendeten \KOMAScript-Klasse \texttt{scrbook} ist die oberste Gliederungsebene,
die in der Bachelorarbeit verwendet werden sollte, das \texttt{\textbackslash chapter}.

Ein Kapitel sollte erst dann in tiefere Gliederungsebenen unterteilt werden, wenn es auch wirklich etwas zu unterteilen gibt. Es sollte keine Kapitel mit nur einem Unterkapitel (\texttt{\textbackslash section}) geben.

In dieser Vorlage ist die Tiefe des Inhaltsverzeichnisses auf \texttt{chapter} und \texttt{section} beschränkt. Möchten Sie diese Beschränkung aufheben, entfernen Sie den Befehl
\begin{verbatim}
            \setcounter{tocdepth}{1}
\end{verbatim}
aus der Präambel oder ändern Sie den Zahlenwert entsprechend. Das Inhaltsverzeichnis sollte für eine Bachelorarbeit auf eine Seite passen.
