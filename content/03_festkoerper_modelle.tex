\chapter{Festkörper-Modelle}
\label{ch:festkoerpermodelle}

Um das Verhalten von Elektronen in einem Festkörper zu beschreiben, wird hauptsächlich zwischen drei Ansätzen unterschieden:
\begin{enumerate}
  \item Die Elektronen bewegen sich frei von Wechselwirkungen mit den Atomrümpfen durch den Festkörper.
  \item Die Elektronen verhalten sich quasifrei in einem periodischen Gitterpotential.
  \item Die Elektronen sind aufgrund eines starken Gitterpotentials an den Atomrümpfen lokalisiert und können gegebenenfalls zwischen den Gitterplätzen springen
\end{enumerate}
In diesem Kapitel werden mehrere Festkörper-Modelle, die von dem dritten aufgezählten Ansatz ausgehen, vorgestellt.
Dabei werden nur Nächst-Nachbar-Wechselwirkungen zwischen den Elektronen betrachtet und Wechselwirkungen zwischen den Atomrümpfen des Festkörpers vernachlässigt.

kittel

\section{Das Tight-binding-Modell und der Stromoperator}

Im Tight-Binding-Modell wird ein Festkörpersystem beschrieben, in welchem die Elektronen stark gebunden und lokalisiert sind und abgesehen von der Pauliabstoßung nicht miteinander in Wechselwirkung treten.
Die Dynamik in diesem Modell ist durch diskrete Sprünge der Elektronen zwischen den Gitterplätzen des Systems gegeben. Der Hamiltonoperator lautet in zweiter Quantisierung
\begin{align}
  H_\text{tb} = -J \sum_{i,\sigma} (\underbrace{c_{i+1,\sigma}^{\dag}c_{i,\sigma}^{\phantom{\dag}}}{\alpha} + \underbrace{c_{i,\sigma}^{\dag}c_{i+1,\sigma}^{\phantom{\dag}}}{\beta}).
  \label{eqn:hamiltontb}
\end{align}
Summiert wird dabei über alle Gitterpätze $i$ und Spinausrichtungen $\sigma \in \{ 0,1 \}$ der Elektronen.
Der Term $\alpha$ lässt ein Elektron lässt vom Gitterplatz $i$ zu $i+1$ springen und der Term $\beta$ in die umgekehrte Richtung.
Der Koeffizient $J$ spiegelt die Stärke der Tight-Binding-Wechselwirkung und damit indirekt die Wahrscheinlichkeit für den Sprung eines Elektrons zwischen zwei Gitterplätzen wieder.

Aus der Kontinuitätsgleichung für den vektorartigen, elektrischen Stromdichteoperator $\mathfrak{J}$(footnote) und der Heisenbergschen Bewegungsgleichung(footnote) wird die Relation
\begin{align}
  \symbf{\nabla}\mathfrak{J} = \symup{i} \left[H_\text{tb}, \rho \right]
\end{align}
hergeleitet. Daraus folgt
\begin{align}
  \symbf{\nabla}\mathfrak{J} = \symup{i} J \sum_{k,\sigma}
\end{align}
durch Einsetzen des Hamiltonoperators \eqref{eqn:hamiltontb} und des Ladungsdichteoperators \eqref{eqn:ladungsdichteoperator} und Ausrechnen des Kommutators mit den Kommutatorrelationen \eqref{eqn:kommutatoren}.
Beim Übergang in ein diskretes 1D-Gittersystem mit der Gitterkonstante $a$ ergibt sich daraus der eindimensionale Stromdichteoperator
\begin{align}
  \mathfrak{J} = \symup{i} J a \sum_{k,\sigma}.
\end{align}

schwabl

\section{Das Hubbard-Modell}

Das Hubbard-Modell ist eine spezielle Erweiterung des Tight-Binding-modells, welche die Wechselwirkung berücksichtigt.
Dies ist im Hamiltonoperator durch einen zusätzlichen Term, der eine starke Coulombabstoßung zwischen den Elektronen beschreibt, realisiert.
Im Folgenden wird die Abstoßung in erster Ordnung, also für Elektronen, die sich auf demselben Gitterpatz befinden, betrachtet.
Diese Näherung lässt sich dadurch begründen, dass das Coulombpotential mit $\frac1{r}$ abfällt. Der Hamiltonoperator des Hubbardmodells ist dann in zweiter Quantisierung durch
\begin{align}
  H_\text{hubb} = H_\text{tb} + U\sum_{i} \prod_{\sigma} n_{i,\sigma}.
  \label{eqn:hamiltonhubb}
\end{align}
gegeben, wobei erneut über die Gitterplätze $i$ und die Spinausrichtungen $\sigma$ summiert wird.

\section{Der Übergang zum Heisenberg-Austausch-Modell im oberen Potential-Grenzfall des halbgefüllten Hubbard-Modells}

Die Besetzungswahrscheinlichkeit der Zustände im halbgefüllten Hubbardmodell, in denen sich jeweils ein Elektron auf jedem Gitterplatz befindet, steigt mit dem Wechselwirkungs-Potential $U$ an, da
die Doppelbesetzung eines Atoms mit dem Ansteigen des Potentials $U$ energetisch ungünstiger wird. Im Grenzfall $U \to \infty$ besteht der Hilbertraum des Festkörpersystems nur noch aus diesen Zuständen
und die Elektronen können sich nicht mehr zwischen den Gitterplätzen bewegen. Die Dynamik des Systems ist dann nur noch durch virtuelle Prozesse gegeben, in welchen die Spinausrichtung der Elektronen
benachbarter Gitterplätze getauscht oder beibehalten wird. Für einen virtuellen Prozesse zwischen zwei benachbarten Elektronen mit antiparalleler Spinausrichtung, z.B.
\begin{align}
  (\uparrow; \downarrow) &\to (\uparrow \downarrow ; 0) \to (\uparrow; \downarrow),
\end{align}
ergibt sich in zweiter Ordnung Störungstheorie die Energiedifferenz
\begin{align}
  \Delta E_{\uparrow \downarrow} = - \frac{2J^2}{U}.
  \label{eqn:Ediffantipar}
\end{align}
Die erste Ordnung Störungstheorie liefert die Energiedifferenz Null, da im Endzustand eines solchen virtuellen Prozesses erster Ordnung zwei Elektronen auf einem Gitterplatz wären, was durch das Potential
$U \to \infty$ verhindert wird. Höhere Ordnungen
Ein Sprung zwischen zwei benachbarten Elektronen mit paralleler Spinausrichtung ist wegen des Pauliprinzips nicht möglich.
Der Energieunterschied beträgt daher
\begin{align}
  \Delta E_{\uparrow \uparrow} = 0.
  \label{eqn:Ediffpar}
\end{align}
Da die virtuellen Prozesse nur noch von der Spinausrichtung abhängen, wird das System mit dem Heisenberg-Austausch-Modell verglichen.
Der Hamiltonoperator dieses Modells ist durch
\begin{align}
  H_\text{spin} = t \sum_{i} (S_i S_{i+1} + C)
  \label{eqn:hamiltonspincj}
\end{align}
gegeben, wobei in beiden Summen über alle Gitterplätze $i$ summiert wird. Durch das Vergleichen der Energiedifferenzen der unterschiedlichen Spinkonfigurationen in diesem Modell
\begin{align}
  \Delta E_{\uparrow \downarrow} = \bra{\uparrow \downarrow} H_\text{spin} \ket{\uparrow \downarrow} = t\left(C - \frac14\right) \\
  \Delta E_{\uparrow \uparrow} = \bra{\uparrow \uparrow} H_\text{spin} \ket{\uparrow \uparrow} = t\left(C + \frac14\right)
\end{align}
mit den Energiedifferenzen \eqref{eqn:Ediffantipar} und \eqref{eqn:Ediffpar} ergeben sich die Konstanten
\begin{align}
  t = \frac{4 J^2}{U}
  \label{eqn:spint}
\end{align}
und
\begin{align}
  C = -\frac14
\end{align}
für den Hamiltonian \eqref{eqn:hamiltonspincj}.
Im Grenzfall $U \to \infty$ geht der Hamilton-Operator des halbgefüllten Hubbard-Modells \eqref{eqn:hamiltonhubb} in den Hamilton-Operator des Heisenberg-Austausch-Modells
\begin{align}
  H_\text{spin} = \frac{4 J^2}{U} \sum_{i} (S_i S_{i+1} - \frac14)
  \label{eqn:hamiltonspin1}
\end{align}
über.
