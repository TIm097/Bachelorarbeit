\chapter{Isolator-Modelle}
\label{ch:isolatormodelle}

In diesem Kapitel werden verschiedene Arten von theoretischen Isolatoren aufgezählt und mehrere Isolator-Modelle vorgestellt.
Die Modelle werden dabei auf Nächst-Nachbar-Wechselwirkungen und eindimensionale Gittersysteme mit periodischen Randbedingungen reduziert.
Festkörper können anhand ihrer elektrischen Leitfähigkeit in Metalle und Isolatoren unterteilt werden,
wobei der physikalische Unterschied in der Bandstruktur der Elektronendispersion liegt.
In Metallen ist das Leitungsband bei beliebigen Temperaturen teilweise gefüllt, sodass die Elektronen theoretisch mit beliebig kleinen Energien angeregt werden können.
Daher ist der Realteil der Leitfähigkeit endlich und es kann ein Stromfluss generiert werden.
Bei Isolatoren befindet sich die Fermienergie $\epsilon_F$ in einer Bandlücke der Bandstruktur, sodass das unterste Leitungsband am Temperaturnullpunkt unbesetzt ist.
Der Realteil der Leitfähigkeit von Isolatoren ist somit gering, da für die Anregung eines Elektrons mindestens die Energie der Bandlücke aufgebracht werden muss,
und verschwindet für $T \to 0$. Im folgenden Abschnitt werden verschiedene Ursachen für die Enstehung eines leeren Leitungsbandes aufgezählt und damit
theoretische Isolatoren klassifiziert. \cite{czycholl,gebhard}

\section{Klassifizierung von Isolatoren}
\label{sec:klassifizierung}

Ein Metall-Isolator-Übergang kann durch mehrere Phänomene, die den Elektronentransport im Festkörper beeinflussen, auftreten.
In diesem Abschnitt werden verschiedene theoretische Isolatoren vorgestellt, die ihre Isolator-Eigenschaft jeweils durch eines dieser Phänomene erhalten.
Bei einem Band-Isolator verursacht die direkte Wechselwirkung zwischen den Elektronen und dem periodischen Gitterpotential eine Bandlücke und ein leeres Leitungsband.
Dabei muss die Anzahl an Elektronen pro Elementarzelle gerade sein, da nur in diesem Fall ein leeres Leitungsband entstehen kann.
Der Peierls-Isolator erhält seine Isolator-Eigenschaft durch den Einfluss eines effektiven Gitterpotentials, welches aus der Bildung von statischen Gitterverzerrungen
und damit verbundenen Ladungsdichtewellen durch die Wechselwirkung zwischen Elektronen und Gitter entsteht.
Beim Anderson-Isolator liegt die Ursache für die geringe Leitfähigkeit in der Wechselwirkung zwischen Elektronen und Gitterfehlern.
Die Unordnung im Gittersystem, die durch die Gitterfehler hervorgerufen wird, führt zu einer Lokalisierung der Elektronen und somit zu einem Metall-Isolator-Übergang.
Der Mott-Isolator ist durch eine nicht vernachlässigbare Coulomb-Wechselwirkung der Elektronen untereinander charakterisiert.
Dabei wird zwischen dem Mott-Hubbard-Isolator, bei welchem die Isolator-Eigenschaft aus der Ladungskorrelation resultiert,
und dem Mott-Heisenberg-Isolator, bei welchem die Spinkorrelation für die Isolator-Eigenschaft ausschlaggebend ist, unterschieden.
Im folgenden Abschnitt wird ein grundlegendes Festkörper-Modell vorgestellt, das zur Beschreibung der aufgezählten Isolatoren jeweils unterschiedlich erweitert werden kann.
\cite{czycholl,gebhard}

\section{Das Tight-Binding-Modell}

Um Aussagen über Bandstrukturen von Festkörpern zu machen, werden verschiedene Ansätze für das Verhalten der Elektronen im Gitterpotential gemacht.
Während die Elektronen im Bloch-Theorem als quasifrei in einem schwachen periodischen Gitterpotential angenommen werden, so werden sie
im Tight-Binding-Modell als gebunden und an den diskreten Gitterplätzen lokalisiert angenommen. Das Tight-Binding-Modell ist unter anderem Grundlage für das Hubbard-Modell,
das zur Beschreibung von Hubbard-Mott-Isolatoren dient, und wird deshalb im Folgenden kurz beschrieben.

Mit der Annahme einer starken Bindung fallen die Wellenfunktionen der Elektronen eines isolierten Atoms signifikant mit mit dem Abstand zum Kern ab.
Wenn der Abstand zwischen den Atomen eines Festkörpers hinreichend groß ist, überlappen sich die Wellenfunktionen der Elektronen eines Atoms näherungsweise nicht mit
den Coulomb-Potentialen der umgebenden Atome. Somit sind die Eigenzustände und -energien des atomaren Problems auch Eigenzustände und -energien des gesamten Festkörpersystems.
Die Dynamik ist dann durch diskrete Sprünge der Elektronen zwischen den Gitterplätzen, hervorgerufen durch den Tunneleffekt, gegeben.
Ausgehend von diesen Annahmen, lautet der Hamiltonoperator des Tight-Binding-Modells in zweiter Quantisierung
\begin{align}
  H_\text{Tb} = -J \sum_{i,\sigma} (c_{i+1,\sigma}^{\dag}c_{i,\sigma}^{\phantom{\dag}} + c_{i,\sigma}^{\dag}c_{i+1,\sigma}^{\phantom{\dag}}).
  \label{eqn:hamiltontb}
\end{align}
Summiert wird über alle Gitterplätze $i$ und Spinausrichtungen $\sigma \in \{ \uparrow,\downarrow \}$ der Elektronen.
Der Koeffizient $J$ gibt die Stärke der Tight-Binding-Wechselwirkung an und ist daher mit der Tunnelwahrscheinlichkeit eines Elektrons zwischen zwei Gitterplätzen korreliert. \cite{czycholl}

\section{Das Hubbard-Modell und der Hubbard-Mott-Isolator}
\label{sec:hubbardmodell}

In diesem Abschnitt wird zunächst das Zustandekommen des Metall-Mott-Hubbard-Isolator-Übergangs für ein eindimensionales Gittersystem erklärt und anschließend
eine mathematische Beschreibung des Hubbard-Modells angegeben.
Der Tight-Binding-Ansatz liefert für ein eindimensionales Gittersystem mit der Gitterkonstante $a$ die Dispersionsrelation
\begin{align}
  E(k) = -2J \cos \left(ka \right).
  \label{eqn:dispersion}
\end{align}
Diese beschreibt genau ein Energieband mit der zur Tight-Binding-Stärke korrelierten Bandbreite $W = 4J$.
Um die starke Elektron-Elektron-Wechselwirkung, die im Mott-Isolator angenommen wird, zu realisieren, wird der Hubbard-Parameter $U$, welcher die Energiedifferenz
zwischen einem doppelt besetzten Gitterplatz $(\uparrow \downarrow;0)$ und zwei einzeln besetzten Gitterplätzen $(\uparrow; \downarrow)$ angibt, eingeführt.
Es kann gezeigt werden, dass sich das Energieband \eqref{eqn:dispersion} für einen endlichen Hubbard-Parameter $U$ in zwei Energiebänder aufspaltet. Diese überlappen
für $U \ll W$ und driften für einen ansteigenden Hubbard-Parameter $U$ auseinander.
Etwa bei $U=W$ entsteht eine Lücke für Ladungsanregungen, sodass ein Metall-Mott-Hubbard-Isolator-Übergang stattfindet.

Um den Hamiltonoperator des Hubbard-Modells aufzustellen, wird der Hamiltonoperator des Tight-Binding-Modells wie in der vorgestellten Herleitung um einen
Potentialterm, der abhängig von $U$ ist, erweitert. Da der Potentialterm genau dann von Null verschieden ist, wenn eine Doppelbesetzung $n_{i,\uparrow} n_{i,\downarrow} = 1$ vorliegt,
ergibt sich
\begin{align}
  H_\text{Hubb} = H_\text{Tb} + U \sum_{i} n_{i,\uparrow} n_{i,\downarrow},
  \label{eqn:hamiltonhubb}
\end{align}
wobei über die Gitterplätze $i$ summiert wird.
Zu diesem Hamiltonoperator folgt mit dem Ansatz aus Abschnitt \ref{sec:stromoperator} für ein Gittersystem aus $N$ Gitterplätzen der
über alle Gitterplätze gemittelte Stromoperator
\begin{align}
  \mathcal{J} = \frac{\symup{i}J}{N} \sum_{i,\sigma} (c_{i+1,\sigma}^{\dag}c_{i,\sigma}^{\phantom{\dag}} - c_{i,\sigma}^{\dag}c_{i+1,\sigma}^{\phantom{\dag}}).
  \label{eqn:stromoperator}
\end{align}
\cite{czycholl,gebhard,uhrig}

\section{Das Heisenberg-Austausch-Modell und der Heisenberg-Mott-Isolator}

In diesem Abschnitt wird zunächst der Übergang des Mott-Hubbard-Isolators in den Mott-Heisenberg-Isolator erklärt
und anschließend das Heisenberg-Austausch-Modell, welches zur Beschreibung eines Mott-Heisenberg-Isolators dient, beschrieben.
Für einen Hubbard-Parameter $U \gg J$ ist jeder Gitterplatz des halbgefüllten Mott-Hubbard-Isolators
im Grundzustand einfach besetzt, da dies energetisch am günstigsten ist.
Die gesamte Dynamik des Systems ist dann durch eine Spin-Austausch-Wechselwirkung der Elektronen gegeben und
der Mott-Hubbard-Isolator geht in den Mott-Heisenberg-Isolator über.

Im Mott-Heisenberg-Isolator können zwei Elektronen benachbarter Gitterplätze ihre Spinausrichtungen, wenn diese antiparallel zueinander stehen, in virtuellen Prozessen austauschen.
Der Tight-Binding-Term im Hamiltonoperator des Hubbard-Modells \eqref{eqn:hamiltonhubb} wird
als Störung behandelt. Für die virtuellen Austausch-Prozesse zwischen zwei benachbarten Elektronen mit antiparalleler Spinausrichtung ergeben sich
in zweiter Ordnung Störungstheorie die Energiedifferenzen
\begin{align}
  \Delta E_{\uparrow \downarrow} = - \frac{2J^2}{U}.
  \label{eqn:Ediffantipar}
\end{align}
Ein Sprung zwischen zwei benachbarten Elektronen mit paralleler Spinausrichtung ist aufgrund des Pauliprinzips nicht möglich, weshalb der Energieunterschied
\begin{align}
  \Delta E_{\uparrow \uparrow} = 0.
  \label{eqn:Ediffpar}
\end{align}
beträgt. Um den Mott-Heisenberg-Isolator mathematisch zu beschreiben, wird das Hubbard-Modell für $U \gg J$
mit dem Heisenberg-Austausch-Modell verglichen. Der zugehörige Hamiltonoperator lautet
\begin{align}
  H_\text{Spin} = t \sum_{i} (\symbf{S}_i \symbf{S}_{i+1} + C),
  \label{eqn:hamiltonspincj}
\end{align}
wobei die Summe über alle Gitterplätze läuft. Durch den Vergleich der Relationen \eqref{eqn:Ediffantipar} und \eqref{eqn:Ediffpar}
mit den Energiedifferenzen der unterschiedlichen Spinkonfigurationen im Heisenberg-Austausch-Modell
ergeben sich der Heisenberg-Parameter
\begin{align}
  t = \frac{4 J^2}{U}
  \label{eqn:spint}
\end{align}
und die Konstante $C = -\tfrac14$.
Daraus und aus der Darstellung des $\symbf{S}^2$-Operators in zweiter Quantisierung \eqref{eqn:squadoperatorviel} folgt die Darstellung des Hamiltonoperators zur Beschreibung eines
Mott-Heisenberg-Isolators in zweiter Quantisierung
\begin{align}
  \begin{split}
    H_\text{Spin} = \, \, \, & \frac{t}{4} \left(\sum_{i} \left(n_{\uparrow,i} - n_{\downarrow,i}\right)\right)^2 \\
    & + \frac{t}{2} \sum_{i} \left( c_{\uparrow,i}^\dag c_{\downarrow,i}^{\phantom{\dag}} \cdot c_{\downarrow,i+1}^\dag c_{\uparrow,i+1}^{\phantom{\dag}} +
    c_{\downarrow,i}^\dag c_{\uparrow,i}^{\phantom{\dag}} \cdot c_{\uparrow,i+1}^\dag c_{\downarrow,i+1}^{\phantom{\dag}} - \frac12 \right).
  \end{split}
  \label{eqn:hamiltonspin2}
\end{align}
\cite{gebhard,uhrig}
