\chapter{Der inverse Faraday Effekt}
\label{ch:ife}

In diesem Kapitel wird der inverse Faraday Effekt in Metallen und Isolatoren behandelt.
Der inverse Faraday-Effekt (im Folgenden mit IFE abgekürzt) beschreibt die Anregung eines Kreisstroms in einem Material durch das Bestrahlen mit zirkular polarisiertem Licht.
Der angeregte Kreisstrom induziert dabei eine Magnetisierung, die gegebenenfalls mit geordneten magnetischen Momenten im Material wechselwirkt.
In den folgenden Abschnitten wird ein Ausdruck für diese durch den IFE angeregte Magnetisierung in Metallen und Isolatoren hergeleitet.
Dabei werden der Magnetfeld-Anteil und die räumliche Oszillation des zirkular polarisierten Lichts vernachlässigt und
quantenmechanische Effekte der Elektronen im Festkörper ignoriert. \cite{hertel} \cite{jäckl}

\section{Der inverse Faraday Effekt in Metallen}

Die Grundannahme für die Herleitung des IFE in Metallen ist, dass sich die Elektronen wie ein wechselwirkungsfreies Plasma mit der
Dichte $n(\vec{x},t)$ und der Geschwindigkeit $\vec{v}(\vec{x},t)$ verhalten. Für die elektrische Stromdichte gilt
\begin{align}
  \vec{j} = n(\vec{x},t) \cdot \vec{v}(\vec{x},t) = \sigma \cdot \delta \vec{E}(\vec{x},t),
\end{align}
wobei $\sigma$ die elektrische Leitfähigkeit des Plasmas ist und
\begin{align}
  \delta \vec{E}(t) = \vec{E}_0 \symup{e}^{-\symup{i}\omega t}
\end{align}
das zeitlich oszillierende elektrische Feld des polarisierten Lichts mit der Amplitude $\vec{E}_0$ und der Kreisfrequenz $\omega$.
Ausgehend von der Kontinuitätsgleichung
\begin{align}
  \frac{\partial n(\vec{x},t)}{\partial t} + \nabla \left[n(\vec{x},t) \cdot \vec{v}(\vec{x},t)\right] = 0
\end{align}
wird durch Separation der Zeitskalen die Magnetisierung
\begin{align}
  \vec{M} = -\frac{\symup{i}}{4 \langle n \rangle \omega} \left( \sigma^* \vec{E}_0^* \times \sigma \vec{E}_0 \right)
  \label{eqn:magnetisierung}
\end{align}
hergeleitet, wobei $\langle n \rangle$ der zeitliche Mittelwert der Ladungsdichte ist. Durch Einsetzen der elektrischen
Leitfähigkeit für das Elektronenplasma
\begin{align}
  \sigma = \frac{i \langle n \rangle}{\omega}
\end{align}
folgt die im Metall induzierte Magnetisierung durch rechts- bzw. linkszirkular polarisiertes Licht, welches sich in z-Richtung ausbreitet,
\begin{align}
  \vec{M}_\text{Metall} = \pm \frac{\langle n \rangle}{4 \omega^3} \left| \vec{E}_0 \right|^2 \vec{e}_z.
  \label{eqn:metmag}
\end{align}
Nach dieser Gleichung können die magnetischen Momente in beispielsweise ferro- oder ferri-magnetischen Metallen präzise
mit einem zirkular polarisierten Laser angeregt und unter bestimmten Umständen sogar permanent manipuliert werden. \cite{hertel}

\section{Der inverse Faraday Effekt in Isolatoren}
\label{sec:ifeiso}

Bei der Herleitung des IFE in Metallen ist die einzige materialspezifische Größe, welche eingesetzt wird,
die Leitfähigkeit des Elektronenplasmas. Um eine Abschätzung für die Abhängigkeiten des IFE in einem Mott-Isolator zu generieren,
wird die elektrische Leitfähigkeit eines Isolators\cite{fließbach}
\begin{align}
  \sigma = \symup{i} \omega \mathcal{C}
\end{align}
in die Magnetisierung \eqref{eqn:magnetisierung} eingesetzt, wobei $\mathcal{C}$ eine Konstante ist, die von den Dielektrizitäts-Eigenschaften des Materials abhängt.
Daraus ergibt sich die induzierte Magnetisierung durch rechts- bzw. linkszirkular polarisiertes Licht, welches sich in z-Richtung ausbreitet,
\begin{align}
  \vec{M}_\text{Ioslator} = \pm \frac{\omega \mathcal{C}^2}{4 \langle n \rangle} \left| \vec{E}_0 \right|^2 \vec{e}_z.
  \label{eqn:isomag}
\end{align}

Aus diesem Kapitel tritt hervor, dass durch den IFE sowohl in einem Metall, als auch in einem Isolator eine endliche Magnetisierung angeregt werden kann.
Für ferrimagnetische Isolatoren ist dies auch experimentell bestätigt und zur Erzeugung von Spinwellen ausgenutzt worden\cite{jäckl}.
Mit den theoretischen Grundlagen aus Kapitel \ref{ch:theorie} und den vorgestellten Isolator-Modellen aus Kapitel \ref{ch:isolatormodelle}
wird im folgenden Kapitel \ref{ch:ergebnisse} ein Mott-Hubbard-Isolator modelliert und untersucht.
Daraufhin wird zur Untersuchung des IFE in diesem Isolator ein rotierendes E-Feld zur Modellierung von zirkular polarisiertem Licht eingeschaltet und
der Erwartungswert für den angeregten Kreisstrom berechnet. Dieser wird in Abhängigkeit von der Amplitude und der Frequenz des E-Felds aufgetragen
und mit der vorausgesagten induzierten Magnetisierung \eqref{eqn:isomag} verglichen. \cite{hertel}
