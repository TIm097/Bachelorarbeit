\chapter{Isolatoren}\label{make}

Diese Vorlage ist auf die Kompilierung mit \texttt{lualatex} ausgelegt.
Als Dokumentenklasse  wird die \KOMAScript\-Klasse \texttt{scrbook} verwendet.
Falls Sie Änderungen am Layout vornehmen möchten, lesen Sie die \KOMAScript-Dokumentation: \cite{koma}.

Eine umfangreiche Einführung in die moderne Verwendung von \LaTeX{} gibt es hier: \cite{toolbox}, lesenswert ist außerdem das \LaTeX-Tabu: \cite{l2tabu}

Um dieses Dokument vollständig zu erstellen sind maximal vier Programmläufe nötig:
\begin{enumerate}[nosep]
    \item \texttt{lualatex BachelorArbeit.tex}
    \item \texttt{biber BachelorArbeit.bcf}
    \item \texttt{lualatex BachelorArbeit.tex}
    \item \texttt{lualatex BachelorArbeit.tex}
\end{enumerate}

Beim ersten Lauf des \LaTeX-Compilers werden die Kapitel, Links und zitierten Bibliographieeinträge in Hilfsdateien geschrieben.

Dann ist ein Lauf des Programms \texttt{biber} nötig, welches die benötigten Einträge aus der Hilfsdatei einliest, die Einträge aus der \texttt{.bib} Datei einliest, sortiert und formatiert und in eine weitere Hilfsdatei schreibt.

Beim nächsten \LaTeX-Lauf werden dann diese Hilfsdateien eingelesen und Literatur- und Inhaltsverzeichnis erstellt.

Manchmal ist ein vierter Lauf nötig, falls sich durch das einfügen des Literaturverzeichnisses Seitenzahlen verändert haben.

Das Tool \texttt{latexmk} übernimmt dies mit nur einem Programmaufruf und
führ nur so viele Aufrufe durch, wie nötig sind.

\texttt{latexmk --lualatex BachelorArbeit.tex}

Eine gute Option ist es, den \LaTeX{} Output in einem anderen
Ordner zu erzeugen, dies ist mit der \texttt{--output-directory} Option möglich:

\texttt{latexmk --output-directory=build --lualatex BachelorArbeit.tex}


\section{Erstellen des Ausgabedokuments mit Make}

Für diese Vorlage wird ein Makefile zur Verfügung gestellt, welches automatisch alle Schritte ausführt, die für das fertige Dokument nötig sind.
Die Ausgabe erfolgt dabei in den Unterordner \texttt{build/}.
Make prüft, ob die Quelldateien verändert wurden, falls nicht, werden auch keine Befehle ausgeführt.

Falls Sie das Makefile benutzen möchten, sollten Sie alle Abhängigkeiten eintragen (Eigene Dateien für Kapitel, Plots, etc.).


Download und weitere Informationen zu Make gibt es unter \cite{make}. Die Befehle sind für die Bash ausgelegt.
Wenn Sie sie unter Windows nutzen wollen, benötigen Sie einen Bash-Emulator, wie Git Bash, Download unter \cite{gitbash} möglich.
Wenn Sie Make installiert haben, rufen Sie einfach in der Konsole im Verzeichnis der Arbeit den Befehl \texttt{make}.

\section{Erstellen des Ausgabedokuments mit Texmaker}
\subsection{Einrichten der nötigen Befehle}
Ein beliebter Editor für alle Betriebssysteme ist Texmaker, Download unter \cite{texmaker}.
Damit Texmaker das Dokument korrekt kompiliert, fügen sie einen benutzerdefinierten Befehl hinzu:
\begin{enumerate}[nosep]
    \item Klicken sie oben in der Menüleiste auf \emph{Benutzer/in}
    \item Klick auf \emph{Eigene Befehle}
    \item Klich auf \emph{Eigene Befehle editieren}, dort können Sie bis zu 5 eigene Befehle definieren
    \item Geben Sie dem Befehl unter \emph{Menüeintrag} einen Namen und tragen sie folgende Befehle in das Befehlsfeld ein: \\
      \verb+latexmk --lualatex --interaction=batchmode --halt-on-error %.tex |+ \\
    \item Bestätigen Sie mit \emph{OK}
\end{enumerate}

In Abbildung \ref{fig:texmaker} ist ein Screenshot des Befehlsmenü gezeigt. Ihren Befehl können Sie nun im Drop-Down-Menü zum
Kompilieren des Dokuments auswählen und mit einem Klick auf den Pfeil starten.

\subsection{Aufräumen}

Nach einem \LaTeX-Fehler ist es oft notwendig, die erstellten Hilfsdateien zu löschen.
Klicken Sie hierzu auf \emph{Werkzeuge}→\emph{Aufräumen}.


\chapter{\LaTeX-Grundlagen}

Bitte beachten Sie beim Schreiben der Arbeit folgende Konventionen bzw. Grundlagen:

\begin{itemize}
    \item \textbf{Abschnitte und Zeilenumbrüche} \\
        Es sollten im Fließtext keine Zeilenumbrüche mit \textbackslash\textbackslash \ erzwungen werden.
        Schreiben Sie höchsten einen Satz in eine Code-Zeile.
        Absätze werden im Code mit einer Leerzeile markiert und dann entsprechend der Einstellung von \texttt{parskip} in der Dokumentenklasse gesetzt.
    \item \textbf{Kursiv/Aufrecht} \\
        \begin{itemize}
            \item Variablen und physikalische Größen werden kursiv gesetzt.
            \item Einheiten werden immer aufrecht und mit einem halben Leerzeichen Abstand zur Zahl gesetzt. Nutzen Sie \texttt{siunitx}!
            \item Mathematische Konstanten und Funktionen werden ebenfalls aufrecht gesetzt. Zum Beispiel die Eulersche Zahl e, das imaginäre i und das infinitesimale d.
                Im Mathematikmodus können Sie dies mit dem Befehl \verb_\mathrm{}_ erreichen. Für die Funktionen stellt \LaTeX \ Befehle bereit, z.B. \verb+\arccos+.
            \item Integrand und ein $\mathrm{d}x$ sollten ebenfalls durch ein kleines Leerzeichen (\verb+\,+) getrennt werden.
        \end{itemize}



\end{itemize}

\section{Zahlen und Einheiten}

Jede Zahl, jede Einheit und jede Zahl mit Einheit sollte mit Hilfe der in dem Paket \texttt{siunitx} zur Verfügung gestellten Befehle gesetzt werden.
Grundsätzlich gilt: Einheiten werden aufrecht gesetzt und haben ein kleines Leerzeichen (\verb+\,+) Abstand zu ihrer Zahl.
Werden Fließkommazahlen ohne \texttt{siunitx} gesetzt, entsteht ein hässlicher Leerraum zwischen Komma und erster Nachkommastelle, da \LaTeX \ das Komma nicht als Dezimaltrennzeichen, sondern als Satzzeichen interpretiert.

Das Paket wurde mit deutschen Spracheinstellungen (also mit Komma als Dezimaltrennzeichen und $\cdot$ zwischen Zahl und Zehnerpotenz) geladen, sowie mit den Einstellungen, dass die Standardabweichung stets durch $\pm$ abgetrennt wird und Einheiten falls nötig als Brüche ausgegeben werden.

\begin{table}
    \centering
    \caption{Beispiele für siunitx}
    \label{tab:si}
    \begin{tabular}{l r}
        \toprule
        Befehl     &   Ergebnis \\
        \midrule
        \verb+\num{1.2345}+ & \num{1.2345} \\
        \verb+\num{1.2e3}+ & \num{1.2e3} \\
        \verb_\num{1.2 +- 0.2}_ & \num{1.2+-0.2} \\
        \verb+\num{10000}+ & \num{10000} \\
        \verb+\si{\meter\per\second}+ & \si{\meter\per\second} \\
        \verb+\SI{1.2(1)}{\micro\ampere}+ & \SI{1.2(1)}{\micro\ampere} \\
        \verb+\SI{1.2\pm0.1e3}{\kilo\gram\per\cubic\meter}+ & \SI{1.2\pm0.1e3}{\kilo\gram\per\cubic\meter} \\
        \bottomrule
    \end{tabular}
\end{table}

Das Paket stellt unter anderem die drei wichtigen Befehle
\begin{itemize}
    \item \texttt{\textbackslash num\{Zahl\}},
    \item \texttt{\textbackslash si\{Einheit\}} und
    \item \texttt{\textbackslash SI\{Zahl\}\{Einheit\}}
\end{itemize}
zur Verfügung.
Diese Befehle sollten stets genutzt werden, wenn Zahlen angegeben werden.
Sie funktionieren sowohl im Text- als auch im Mathematikmodus.
In Tabelle \ref{tab:si} sind einige Beispiele aufgetragen. Bitte lesen Sie die Dokumentation \cite{siunitx}.

\section{Das Literaturverzeichnis}

Das Literaturverzeichnis wird mit Hilfe von BibLaTeX und biber erstellt.
Tragen Sie alle ihre Quellen in die Datei \texttt{references.bib} ein, Sie enthält bereits
einige Beispiele. Für weitere Informationen lesen Sie bitte die Dokumentation \cite{biblatex}.

Im Text können Sie mit \verb_\cite{kürzel}_ zitieren. Seitenzahlen geben Sie in eckigen Klammern an:
\verb_\cite[10]{kürzel}_.

Das Literaturverzeichnis ist so eingestellt, dass es Ihre Quellen in alphabetischer Reihenfolge nach Autoren nummeriert.
Möchten Sie das Literaturverzeichnis nach der Reihenfolge des Auftauchens im Text sortieren, fügen sie die Paktetoption \texttt{sorting=none} beim Laden
des BibLaTeX-Pakets hinzu.

Den Zitier- und Bibliographie-Stil geben sie mit der Option \texttt{style=Stil} an. Die beiden gebräuchlisten Stile sind \texttt{numeric} und \texttt{alphabetic}.
Bei \texttt{numeric} werden die Quellen durchnummeriert, bei \texttt{alphabetic} wird ein Buchstabenkürzel aus Autor(en)-Name(n) und Jahr verwendet.
Für weitere Stile konsultieren Sie bitte die Dokumentation: \cite{biblatex}.

Ein Beispiel für das Zitieren eines Buches lautet so \cite{handbook_adhesives},
wissenschaftliche Artikel hingegen werden so \cite{einstein} zitiert.

Damit das Literaturverzeichnis erstellt wird, ist ein Aufruf von \texttt{biber} nach einem ersten kompilieren mit \texttt{lualatex} nötig.
Danach muss das Dokument erneut mit \texttt{lualatex} kompiliert werden.

Zum korrekten Kompilieren des Dokuments siehe Kapitel \ref{make}.
