\chapter{Anhang}
\label{ch:anhang}

\section{Basis-Zustände und Matrixdarstellungen von Operatoren}
\setcounter{MaxMatrixCols}{18}

\begin{table}
  \centering
  \caption{Darstellung der vier Basis-Zustände des 4N1E-Systems aus Abschnitt \ref{sec:untersuchungtb} mit der Vorschrift \eqref{eqn:4N1Ezustandsvorschrift}.}
  \begin{tabular}{S[table-format=1.0] S}
    \toprule
    {$i$} & {$\ket{i}$} \\
    \midrule
    1 & $\ket{1}$ \\
    2 & $\ket{2}$ \\
    3 & $\ket{3}$ \\
    4 & $\ket{4}$ \\
    \bottomrule
  \end{tabular}
  \label{tab:4N1Ezust}
\end{table}
Die Matrixdarstellung des Hamiltonoperators für das 4N1E-System aus Abschnitt \ref{sec:untersuchungtb} zu den Basis-Zuständen aus Tabelle \ref{tab:4N1Ezust} lautet
\begin{align}
  H_{ij} &=
  \begingroup
    \setlength\arraycolsep{2.5pt}
    \begin{pmatrix*}[r]
      0 & -1 & 0 & -1 \\
      -1 & 0 & -1 & 0 \\
      0 & -1 & 0 & -1 \\
      -1 & 0 & -1 & 0
    \end{pmatrix*}_{ij}
  \endgroup
  \label{eqn:4N1Ehamiltonmatrix}
\end{align}
in Einheiten von $J$.

\newpage

\begin{table}
  \centering
  \caption{Darstellung der 36 Zustände des 8N4E-Systems im Hubbard-Modell aus Abschnitt \ref{sec:untersuchunghubb} mit der Vorschrift \eqref{eqn:8N4Ezustandsvorschrift}.}
  \begin{minipage}[t]{0.35\linewidth}
    \begin{tabular}{S[table-format=2.0] S}
      \toprule
      {$i$} & {$\ket{i}$} \\
      \midrule
      1  & $\ket{1,2;1,2}$ \\
      2  & $\ket{1,2;1,3}$ \\
      3  & $\ket{1,2;1,4}$ \\
      4  & $\ket{1,2;2,3}$ \\
      5  & $\ket{1,2;2,4}$ \\
      6  & $\ket{1,2;3,4}$ \\
      7  & $\ket{1,3;1,2}$ \\
      8  & $\ket{1,3;1,3}$ \\
      9  & $\ket{1,3;1,4}$ \\
      10 & $\ket{1,3;2,3}$ \\
      11 & $\ket{1,3;2,4}$ \\
      12 & $\ket{1,3;3,4}$ \\
      13 & $\ket{1,4;1,2}$ \\
      14 & $\ket{1,4;1,3}$ \\
      15 & $\ket{1,4;1,4}$ \\
      16 & $\ket{1,4;2,3}$ \\
      17 & $\ket{1,4;2,4}$ \\
      18 & $\ket{1,4;3,4}$ \\
      \bottomrule
    \end{tabular}
  \end{minipage}
  \begin{minipage}[t]{0.35\linewidth}
    \begin{tabular}{S[table-format=2.0] S}
      \toprule
      {$i$} & {$\ket{i}$} \\
      \midrule
      19 & $\ket{2,3;1,2}$ \\
      20 & $\ket{2,3;1,3}$ \\
      21 & $\ket{2,3;1,4}$ \\
      22 & $\ket{2,3;2,3}$ \\
      23 & $\ket{2,3;2,4}$ \\
      24 & $\ket{2,3;3,4}$ \\
      25 & $\ket{2,4;1,2}$ \\
      26 & $\ket{2,4;1,3}$ \\
      27 & $\ket{2,4;1,4}$ \\
      28 & $\ket{2,4;2,3}$ \\
      29 & $\ket{2,4;2,4}$ \\
      30 & $\ket{2,4;3,4}$ \\
      31 & $\ket{3,4;1,2}$ \\
      32 & $\ket{3,4;1,3}$ \\
      33 & $\ket{3,4;1,4}$ \\
      34 & $\ket{3,4;2,3}$ \\
      35 & $\ket{3,4;2,4}$ \\
      36 & $\ket{3,4;3,4}$ \\
      \bottomrule
    \end{tabular}
  \end{minipage}
  \label{tab:8N4Ehubbzust}
\end{table}
Die Matrixdarstellung des Hamiltonoperators für das 8N4E-System im Hubbard-Modell aus Abschnitt \ref{sec:untersuchunghubb} zu den Basis-Zuständen aus Tabelle \ref{tab:8N4Ehubbzust} lautet
\setcounter{MaxMatrixCols}{18}
\begin{align}
  H_{ij} =
  \begin{pmatrix*}[l]
     \mathcal{A} & \mathcal{B}^T \\
     \mathcal{B} & \mathcal{C}
  \end{pmatrix*}_{ij}
  \label{eqn:8N4Ehubbmatrix}
\end{align}

\newpage

mit
\begin{align*}
  \mathcal{A} & =
  \begingroup
    \setlength\arraycolsep{2.1pt}
    \begin{pmatrix*}[r]
      2\alpha & -1 &  0 &  0 &  1 &  0 & -1 &  0 &  0 &  0 &  0 &  0 &  0 &  0 &  0 &  0 &  0 &  0 \\
      -1 &  \alpha & -1 & -1 &  0 &  1 &  0 & -1 &  0 &  0 &  0 &  0 &  0 &  0 &  0 &  0 &  0 &  0 \\
      0 & -1 &  \alpha &  0 & -1 &  0 &  0 &  0 & -1 &  0 &  0 &  0 &  0 &  0 &  0 &  0 &  0 &  0 \\
      0 & -1 &  0 &  \alpha & -1 &  0 &  0 &  0 &  0 & -1 &  0 &  0 &  0 &  0 &  0 &  0 &  0 &  0 \\
      1 &  0 & -1 & -1 &  \alpha & -1 &  0 &  0 &  0 &  0 & -1 &  0 &  0 &  0 &  0 &  0 &  0 &  0 \\
      0 &  1 &  0 &  0 & -1 &  0 &  0 &  0 &  0 &  0 &  0 & -1 &  0 &  0 &  0 &  0 &  0 &  0 \\
      -1 &  0 &  0 &  0 &  0 &  0 &  \alpha & -1 &  0 &  0 &  1 &  0 & -1 &  0 &  0 &  0 &  0 &  0 \\
      0 & -1 &  0 &  0 &  0 &  0 & -1 &  2\alpha & -1 & -1 &  0 &  1 &  0 & -1 &  0 &  0 &  0 &  0 \\
      0 &  0 & -1 &  0 &  0 &  0 &  0 & -1 &  \alpha &  0 & -1 &  0 &  0 &  0 & -1 &  0 &  0 &  0 \\
      0 &  0 &  0 & -1 &  0 &  0 &  0 & -1 &  0 &  \alpha & -1 &  0 &  0 &  0 &  0 & -1 &  0 &  0 \\
      0 &  0 &  0 &  0 & -1 &  0 &  1 &  0 & -1 & -1 &  0 & -1 &  0 &  0 &  0 &  0 & -1 &  0 \\
      0 &  0 &  0 &  0 &  0 & -1 &  0 &  1 &  0 &  0 & -1 &  \alpha &  0 &  0 &  0 &  0 &  0 & -1 \\
      0 &  0 &  0 &  0 &  0 &  0 & -1 &  0 &  0 &  0 &  0 &  0 &  \alpha & -1 &  0 &  0 &  1 &  0 \\
      0 &  0 &  0 &  0 &  0 &  0 &  0 & -1 &  0 &  0 &  0 &  0 & -1 &  \alpha & -1 & -1 &  0 &  1 \\
      0 &  0 &  0 &  0 &  0 &  0 &  0 &  0 & -1 &  0 &  0 &  0 &  0 & -1 &  2\alpha &  0 & -1 &  0 \\
      0 &  0 &  0 &  0 &  0 &  0 &  0 &  0 &  0 & -1 &  0 &  0 &  0 & -1 &  0 &  0 & -1 &  0 \\
      0 &  0 &  0 &  0 &  0 &  0 &  0 &  0 &  0 &  0 & -1 &  0 &  1 &  0 & -1 & -1 &  \alpha & -1 \\
      0 &  0 &  0 &  0 &  0 &  0 &  0 &  0 &  0 &  0 &  0 & -1 &  0 &  1 &  0 &  0 & -1 &  \alpha
    \end{pmatrix*}
  \endgroup \\
  \mathcal{B} & =
  \begingroup
    \setlength\arraycolsep{2.1pt}
    \begin{pmatrix*}[r]
      0 & 0 & 0 & 0 & 0 & 0 &-1 & 0 & 0 & 0 & 0 & 0 & 0 & 0 & 0 & 0 & 0 & 0 \\
      0 & 0 & 0 & 0 & 0 & 0 & 0 &-1 & 0 & 0 & 0 & 0 & 0 & 0 & 0 & 0 & 0 & 0 \\
      0 & 0 & 0 & 0 & 0 & 0 & 0 & 0 &-1 & 0 & 0 & 0 & 0 & 0 & 0 & 0 & 0 & 0 \\
      0 & 0 & 0 & 0 & 0 & 0 & 0 & 0 & 0 &-1 & 0 & 0 & 0 & 0 & 0 & 0 & 0 & 0 \\
      0 & 0 & 0 & 0 & 0 & 0 & 0 & 0 & 0 & 0 &-1 & 0 & 0 & 0 & 0 & 0 & 0 & 0 \\
      0 & 0 & 0 & 0 & 0 & 0 & 0 & 0 & 0 & 0 & 0 &-1 & 0 & 0 & 0 & 0 & 0 & 0 \\
      \phantom{-}1 & 0 & 0 & 0 & 0 & 0 & 0 & 0 & 0 & 0 & 0 & 0 &-1 & 0 & 0 & 0 & 0 & 0 \\
      0 & \phantom{-}1 & 0 & 0 & 0 & 0 & 0 & 0 & 0 & 0 & 0 & 0 & 0 &-1 & 0 & 0 & 0 & 0 \\
      0 & 0 & \phantom{-}1 & 0 & 0 & 0 & 0 & 0 & 0 & 0 & 0 & 0 & 0 & 0 &-1 & 0 & 0 & 0 \\
      0 & 0 & 0 & \phantom{-}1 & 0 & 0 & 0 & 0 & 0 & 0 & 0 & 0 & 0 & 0 & 0 &-1 & 0 & 0 \\
      0 & 0 & 0 & 0 & \phantom{-}1 & 0 & 0 & 0 & 0 & 0 & 0 & 0 & 0 & 0 & 0 & 0 &-1 & 0 \\
      0 & 0 & 0 & 0 & 0 & \phantom{-}1 & 0 & 0 & 0 & 0 & 0 & 0 & 0 & 0 & 0 & 0 & 0 &-1 \\
      0 & 0 & 0 & 0 & 0 & 0 & \phantom{-}1 & 0 & 0 & 0 & 0 & 0 & 0 & 0 & 0 & 0 & 0 & 0 \\
      0 & 0 & 0 & 0 & 0 & 0 & 0 & \phantom{-}1 & 0 & 0 & 0 & 0 & 0 & 0 & 0 & 0 & 0 & 0 \\
      0 & 0 & 0 & 0 & 0 & 0 & 0 & 0 & \phantom{-}1 & 0 & 0 & 0 & 0 & 0 & 0 & 0 & 0 & 0 \\
      0 & 0 & 0 & 0 & 0 & 0 & 0 & 0 & 0 & \phantom{-}1 & 0 & 0 & 0 & 0 & 0 & 0 & 0 & 0 \\
      0 & 0 & 0 & 0 & 0 & 0 & 0 & 0 & 0 & 0 & \phantom{-}1 & 0 & 0 & 0 & 0 & 0 & 0 & 0 \\
      0 & 0 & 0 & 0 & 0 & 0 & 0 & 0 & 0 & 0 & 0 & \phantom{-}1 & 0 & 0 & 0 & 0 & 0 & 0
    \end{pmatrix*}
  \endgroup
\end{align*}

\newpage

\begin{align*}
  \mathcal{C} &=
  \begingroup
    \setlength\arraycolsep{2.1pt}
    \begin{pmatrix*}[r]
      \alpha &-1 & 0 & 0 & 1 & 0 &-1 & 0 & 0 & 0 & 0 & 0 & 0 & 0 & 0 & 0 & 0 & 0 \\
      -1 & \alpha &-1 &-1 & 0 & 1 & 0 &-1 & 0 & 0 & 0 & 0 & 0 & 0 & 0 & 0 & 0 & 0 \\
      0 &-1 & 0 & 0 &-1 & 0 & 0 & 0 &-1 & 0 & 0 & 0 & 0 & 0 & 0 & 0 & 0 & 0 \\
      0 &-1 & 0 & 2\alpha&-1 & 0 & 0 & 0 & 0 &-1 & 0 & 0 & 0 & 0 & 0 & 0 & 0 & 0 \\
      1 & 0 &-1 &-1 & \alpha &-1 & 0 & 0 & 0 & 0 &-1 & 0 & 0 & 0 & 0 & 0 & 0 & 0 \\
      0 & 1 & 0 & 0 &-1 & \alpha & 0 & 0 & 0 & 0 & 0 &-1 & 0 & 0 & 0 & 0 & 0 & 0 \\
      -1 & 0 & 0 & 0 & 0 & 0 & \alpha &-1 & 0 & 0 & 1 & 0 &-1 & 0 & 0 & 0 & 0 & 0 \\
      0 &-1 & 0 & 0 & 0 & 0 &-1 & 0 &-1 &-1 & 0 & 1 & 0 &-1 & 0 & 0 & 0 & 0 \\
      0 & 0 &-1 & 0 & 0 & 0 & 0 &-1 & \alpha & 0 &-1 & 0 & 0 & 0 &-1 & 0 & 0 & 0 \\
      0 & 0 & 0 &-1 & 0 & 0 & 0 &-1 & 0 & \alpha &-1 & 0 & 0 & 0 & 0 &-1 & 0 & 0 \\
      0 & 0 & 0 & 0 &-1 & 0 & 1 & 0 &-1 &-1 & 2\alpha&-1 & 0 & 0 & 0 & 0 &-1 & 0 \\
      0 & 0 & 0 & 0 & 0 &-1 & 0 & 1 & 0 & 0 &-1 & \alpha & 0 & 0 & 0 & 0 & 0 &-1 \\
      0 & 0 & 0 & 0 & 0 & 0 &-1 & 0 & 0 & 0 & 0 & 0 & 0 &-1 & 0 & 0 & 1 & 0 \\
      0 & 0 & 0 & 0 & 0 & 0 & 0 &-1 & 0 & 0 & 0 & 0 &-1 & \alpha &-1 &-1 & 0 & 1 \\
      0 & 0 & 0 & 0 & 0 & 0 & 0 & 0 &-1 & 0 & 0 & 0 & 0 &-1 & \alpha & 0 &-1 & 0 \\
      0 & 0 & 0 & 0 & 0 & 0 & 0 & 0 & 0 &-1 & 0 & 0 & 0 &-1 & 0 & \alpha &-1 & 0 \\
      0 & 0 & 0 & 0 & 0 & 0 & 0 & 0 & 0 & 0 &-1 & 0 & 1 & 0 &-1 &-1 & \alpha &-1 \\
      0 & 0 & 0 & 0 & 0 & 0 & 0 & 0 & 0 & 0 & 0 &-1 & 0 & 1 & 0 & 0 &-1 & 2\alpha
    \end{pmatrix*}
  \endgroup
\end{align*}
in Einheiten von $J$ und mit $\alpha = U/J$.

\begin{table}
  \centering
  \caption{Darstellung der sechs Zustände des 8N4E-Systems im Heisenberg-Austausch-Modell aus Abschnitt \ref{sec:untersuchungheis} mit der Vorschrift \eqref{eqn:8N4Ezustandsvorschrift}.}
  \begin{tabular}{S[table-format=1.0] S}
    \toprule
    {$i$} & {$\ket{i}$} \\
    \midrule
    1  & $\ket{1,2;3,4}$ \\
    2  & $\ket{1,3;2,4}$ \\
    3  & $\ket{1,4;3,2}$ \\
    4  & $\ket{3,4;1,2}$ \\
    5  & $\ket{2,4;1,3}$ \\
    6  & $\ket{3,2;1,4}$ \\
    \bottomrule
  \end{tabular}
  \label{tab:8N4Eheiszust}
\end{table}
Die Matrixdarstellung des Hamiltonoperators für das 8N4E-System im Heisenberg-Austausch-Modell aus Abschnitt \ref{sec:untersuchungheis} zu den Basis-Zuständen aus Tabelle \ref{tab:8N4Eheiszust} lautet
\setcounter{MaxMatrixCols}{18}
\begin{align}
  H_{ij} =
  \begin{pmatrix*}[r]
    -1   &  0.5 &  0   &  0   &  0.5 &  0   \\
     0.5 & -2   & -0.5 &  0.5 &  0   & -0.5 \\
     0   & -0.5 & -1   &  0   & -0.5 &  0   \\
     0   &  0.5 &  0   & -1   &  0.5 &  0   \\
     0.5 &  0   & -0.5 &  0.5 & -2   & -0.5 \\
     0   & -0.5 &  0   &  0   & -0.5 & -1
  \end{pmatrix*}_{ij}
  \label{eqn:8N4Eheismatrix}
\end{align}
in Einheiten von $t$.

Die Matrixdarstellung des Feldoperators $H_{\Phi}$ aus Abschnitt \ref{sec:untersuchungife} zu den Basis-Zuständen aus Tabelle \ref{tab:8N4Ehubbzust} lautet
\begin{align}
  H_{ij} = \delta_{ij}
  \begin{pmatrix}
    \vec{\mathcal{\chi}_1} \\
    \vec{\mathcal{\chi}_2} \\
  \end{pmatrix}_{i}
  \label{eqn:ifematrix}
\end{align}
mit
\begin{align*}
  \vec{\mathcal{\chi}_1} & =
  \begin{pmatrix}
    -2 \sin{\omega t} \\
    -\sin{\omega t} \\
    -\cos{\omega t}  -\sin{\omega t} \\
    \cos{\omega t}  -\sin{\omega t} \\
    -\sin{\omega t} \\
    0                                    \\
    -\sin{\omega t} \\
    0                                    \\
    -\cos{\omega t}                    \\
    \cos{\omega t}                    \\
    0                                    \\
    \sin{\omega t} \\
    -\cos{\omega t}  -\sin{\omega t} \\
    -\cos{\omega t}                    \\
    -2 \cos{\omega t}                    \\
    0                                    \\
    -\cos{\omega t}                    \\
    -\cos{\omega t}  +\sin{\omega t} \\
\end{pmatrix}
& \vec{\mathcal{\chi}_1} = &
\begin{pmatrix}
    \cos{\omega t}  -\sin{\omega t} \\
    \cos{\omega t}                    \\
    0                                    \\
    2 \cos{\omega t}                    \\
    \cos{\omega t}                    \\
    \cos{\omega t}  +\sin{\omega t} \\
    -\sin{\omega t} \\
    0                                    \\
    -\cos{\omega t}                    \\
    \cos{\omega t}                    \\
    0                                    \\
    \sin{\omega t} \\
    0                                    \\
    \sin{\omega t} \\
    -\cos{\omega t}  +\sin{\omega t} \\
    \cos{\omega t}  +\sin{\omega t} \\
    \sin{\omega t} \\
    2 \sin{\omega t} \\
  \end{pmatrix}
\end{align*}
in Einheiten von $a A_0 \symup{e}$.

Die Matrixdarstellung des Stromoperators aus Abschnitt \ref{sec:untersuchungife} zu den Basis-Zuständen aus Tabelle \ref{tab:8N4Ehubbzust} lautet
\begin{align}
\mathcal{J}_{ij} =
\begin{pmatrix*}[l]
   \mathcal{U} & -\mathcal{V}^T \\
   \mathcal{V} & \phantom{-}\mathcal{W}
\end{pmatrix*}_{ij}
\label{eqn:strommatrix}
\end{align}
mit

\begin{align*}
  \mathcal{U} &=
  \begingroup
    \setlength\arraycolsep{2.1pt}
    \begin{pmatrix*}[r]
      0 & -1 &  0 &  0 & -1 &  0 & -1 &  0 &  0 &  0 &  0 &  0 &  0 &  0 &  0 &  0 &  0 &  0 \\
      \phantom{-}1 & 0 & -1 & -1 &  0 & -1 &  0 & -1 &  0 &  0 &  0 &  0 &  0 &  0 &  0 &  0 &  0 &  0 \\
      0 &\phantom{-}1 & 0 &  0 & -1 &  0 &  0 &  0 & -1 &  0 &  0 &  0 &  0 &  0 &  0 &  0 &  0 &  0 \\
      0 &\phantom{-}1 & 0 & 0 & -1 &  0 &  0 &  0 &  0 & -1 &  0 &  0 &  0 &  0 &  0 &  0 &  0 &  0 \\
      \phantom{-}1 & 0 &\phantom{-}1 &\phantom{-}1 & 0 & -1 &  0 &  0 &  0 &  0 & -1 &  0 &  0 &  0 &  0 &  0 &  0 &  0 \\
      0 &\phantom{-}1 & 0 & 0 &\phantom{-}1 & 0 &  0 &  0 &  0 &  0 &  0 & -1 &  0 &  0 &  0 &  0 &  0 &  0 \\
      \phantom{-}1 & 0 & 0 & 0 & 0 & 0 & 0 & -1 &  0 &  0 & -1 &  0 & -1 &  0 &  0 &  0 &  0 &  0 \\
      0 &\phantom{-}1 & 0 & 0 & 0 & 0 &\phantom{-}1 & 0 & -1 & -1 &  0 & -1 &  0 & -1 &  0 &  0 &  0 &  0 \\
      0 & 0 &\phantom{-}1 & 0 & 0 & 0 & 0 &\phantom{-}1 & 0 &  0 & -1 &  0 &  0 &  0 & -1 &  0 &  0 &  0 \\
      0 & 0 & 0 &\phantom{-}1 & 0 & 0 & 0 &\phantom{-}1 & 0 & 0 & -1 &  0 &  0 &  0 &  0 & -1 &  0 &  0 \\
      0 & 0 & 0 & 0 &\phantom{-}1 & 0 &\phantom{-}1 & 0 &\phantom{-}1 &\phantom{-}1 & 0 & -1 &  0 &  0 &  0 &  0 & -1 &  0 \\
      0 & 0 & 0 & 0 & 0 &\phantom{-}1 & 0 &\phantom{-}1 & 0 & 0 &\phantom{-}1 & 0 &  0 &  0 &  0 &  0 &  0 & -1 \\
      0 & 0 & 0 & 0 & 0 & 0 &\phantom{-}1 & 0 & 0 & 0 & 0 & 0 & 0 & -1 &  0 &  0 & -1 &  0 \\
      0 & 0 & 0 & 0 & 0 & 0 & 0 &\phantom{-}1 & 0 & 0 & 0 & 0 &\phantom{-}1 & 0 & -1 & -1 &  0 & -1 \\
      0 & 0 & 0 & 0 & 0 & 0 & 0 & 0 &\phantom{-}1 & 0 & 0 & 0 & 0 &\phantom{-}1 & 0 &  0 & -1 &  0 \\
      0 & 0 & 0 & 0 & 0 & 0 & 0 & 0 & 0 &\phantom{-}1 & 0 & 0 & 0 &\phantom{-}1 & 0 & 0 & -1 &  0 \\
      0 & 0 & 0 & 0 & 0 & 0 & 0 & 0 & 0 & 0 &\phantom{-}1 & 0 &\phantom{-}1 & 0 &\phantom{-}1 &\phantom{-}1 & 0 & -1 \\
      0 & 0 & 0 & 0 & 0 & 0 & 0 & 0 & 0 & 0 & 0 &\phantom{-}1 & 0 &\phantom{-}1 & 0 & 0 &\phantom{-}1 & 0
    \end{pmatrix*}
  \endgroup\\
  \mathcal{V} &=
  \begingroup
    \setlength\arraycolsep{2.1pt}
    \begin{pmatrix*}[r]
      0 & 0 & 0 & 0 & 0 & 0 & \phantom{-}1 & 0 & 0 & 0 & 0 & 0 & 0 & 0 & 0 & 0 & 0 & 0 \\
      0 & 0 & 0 & 0 & 0 & 0 & 0 & \phantom{-}1 & 0 & 0 & 0 & 0 & 0 & 0 & 0 & 0 & 0 & 0 \\
      0 & 0 & 0 & 0 & 0 & 0 & 0 & 0 & \phantom{-}1 & 0 & 0 & 0 & 0 & 0 & 0 & 0 & 0 & 0 \\
      0 & 0 & 0 & 0 & 0 & 0 & 0 & 0 & 0 & \phantom{-}1 & 0 & 0 & 0 & 0 & 0 & 0 & 0 & 0 \\
      0 & 0 & 0 & 0 & 0 & 0 & 0 & 0 & 0 & 0 & \phantom{-}1 & 0 & 0 & 0 & 0 & 0 & 0 & 0 \\
      0 & 0 & 0 & 0 & 0 & 0 & 0 & 0 & 0 & 0 & 0 & \phantom{-}1 & 0 & 0 & 0 & 0 & 0 & 0 \\
      \phantom{-}1 & 0 & 0 & 0 & 0 & 0 & 0 & 0 & 0 & 0 & 0 & 0 & \phantom{-}1 & 0 & 0 & 0 & 0 & 0 \\
      0 & \phantom{-}1 & 0 & 0 & 0 & 0 & 0 & 0 & 0 & 0 & 0 & 0 & 0 & \phantom{-}1 & 0 & 0 & 0 & 0 \\
      0 & 0 & \phantom{-}1 & 0 & 0 & 0 & 0 & 0 & 0 & 0 & 0 & 0 & 0 & 0 & \phantom{-}1 & 0 & 0 & 0 \\
      0 & 0 & 0 & \phantom{-}1 & 0 & 0 & 0 & 0 & 0 & 0 & 0 & 0 & 0 & 0 & 0 & \phantom{-}1 & 0 & 0 \\
      0 & 0 & 0 & 0 & \phantom{-}1 & 0 & 0 & 0 & 0 & 0 & 0 & 0 & 0 & 0 & 0 & 0 & \phantom{-}1 & 0 \\
      0 & 0 & 0 & 0 & 0 & \phantom{-}1 & 0 & 0 & 0 & 0 & 0 & 0 & 0 & 0 & 0 & 0 & 0 & \phantom{-}1 \\
      0 & 0 & 0 & 0 & 0 & 0 & \phantom{-}1 & 0 & 0 & 0 & 0 & 0 & 0 & 0 & 0 & 0 & 0 & 0 \\
      0 & 0 & 0 & 0 & 0 & 0 & 0 & \phantom{-}1 & 0 & 0 & 0 & 0 & 0 & 0 & 0 & 0 & 0 & 0 \\
      0 & 0 & 0 & 0 & 0 & 0 & 0 & 0 & \phantom{-}1 & 0 & 0 & 0 & 0 & 0 & 0 & 0 & 0 & 0 \\
      0 & 0 & 0 & 0 & 0 & 0 & 0 & 0 & 0 & \phantom{-}1 & 0 & 0 & 0 & 0 & 0 & 0 & 0 & 0 \\
      0 & 0 & 0 & 0 & 0 & 0 & 0 & 0 & 0 & 0 & \phantom{-}1 & 0 & 0 & 0 & 0 & 0 & 0 & 0 \\
      0 & 0 & 0 & 0 & 0 & 0 & 0 & 0 & 0 & 0 & 0 & \phantom{-}1 & 0 & 0 & 0 & 0 & 0 & 0 \\
    \end{pmatrix*}
  \endgroup
\end{align*}
\begin{align*}
  \mathcal{W} &=
  \begingroup
    \setlength\arraycolsep{2.1pt}
    \begin{pmatrix*}[r]
      0& -1 & 0 & 0& -1 & 0& -1 & 0 & 0 & 0 & 0 & 0 & 0 & 0 & 0 & 0 & 0 & 0 \\
      \phantom{-}1 & 0& -1& -1 & 0& -1 & 0& -1 & 0 & 0 & 0 & 0 & 0 & 0 & 0 & 0 & 0 & 0 \\
      0 &\phantom{-}1 & 0 & 0& -1 & 0 & 0 & 0& -1 & 0 & 0 & 0 & 0 & 0 & 0 & 0 & 0 & 0 \\
      0 &\phantom{-}1 & 0 & 0& -1 & 0 & 0 & 0 & 0& -1 & 0 & 0 & 0 & 0 & 0 & 0 & 0 & 0 \\
      \phantom{-}1 & 0 &\phantom{-}1 &\phantom{-}1 & 0& -1 & 0 & 0 & 0 & 0& -1 & 0 & 0 & 0 & 0 & 0 & 0 & 0 \\
      0 &\phantom{-}1 & 0 & 0 &\phantom{-}1 & 0 & 0 & 0 & 0 & 0 & 0& -1 & 0 & 0 & 0 & 0 & 0 & 0 \\
      \phantom{-}1 & 0 & 0 & 0 & 0 & 0 & 0& -1 & 0 & 0& -1 & 0& -1 & 0 & 0 & 0 & 0 & 0 \\
      0 &\phantom{-}1 & 0 & 0 & 0 & 0 &\phantom{-}1 & 0& -1& -1 & 0& -1 & 0& -1 & 0 & 0 & 0 & 0 \\
      0 & 0 &\phantom{-}1 & 0 & 0 & 0 & 0 &\phantom{-}1 & 0 & 0& -1 & 0 & 0 & 0& -1 & 0 & 0 & 0 \\
      0 & 0 & 0 &\phantom{-}1 & 0 & 0 & 0 &\phantom{-}1 & 0 & 0& -1 & 0 & 0 & 0 & 0& -1 & 0 & 0 \\
      0 & 0 & 0 & 0 &\phantom{-}1 & 0 &\phantom{-}1 & 0 &\phantom{-}1 &\phantom{-}1 & 0& -1 & 0 & 0 & 0 & 0& -1 & 0 \\
      0 & 0 & 0 & 0 & 0 &\phantom{-}1 & 0 &\phantom{-}1 & 0 & 0 &\phantom{-}1 & 0 & 0 & 0 & 0 & 0 & 0& -1 \\
      0 & 0 & 0 & 0 & 0 & 0 &\phantom{-}1 & 0 & 0 & 0 & 0 & 0 & 0& -1 & 0 & 0& -1 & 0 \\
      0 & 0 & 0 & 0 & 0 & 0 & 0 &\phantom{-}1 & 0 & 0 & 0 & 0 &\phantom{-}1 & 0& -1& -1 & 0& -1 \\
      0 & 0 & 0 & 0 & 0 & 0 & 0 & 0 &\phantom{-}1 & 0 & 0 & 0 & 0 &\phantom{-}1 & 0 & 0& -1 & 0 \\
      0 & 0 & 0 & 0 & 0 & 0 & 0 & 0 & 0 &\phantom{-}1 & 0 & 0 & 0 &\phantom{-}1 & 0 & 0& -1 & 0 \\
      0 & 0 & 0 & 0 & 0 & 0 & 0 & 0 & 0 & 0 &\phantom{-}1 & 0 &\phantom{-}1 & 0 &\phantom{-}1 &\phantom{-}1 & 0& -1 \\
      0 & 0 & 0 & 0 & 0 & 0 & 0 & 0 & 0 & 0 & 0 &\phantom{-}1 & 0 &\phantom{-}1 & 0 & 0 &\phantom{-}1 & 0 \\
    \end{pmatrix*}
  \endgroup
\end{align*}
in Einheiten von $\frac{\symup{i} J \symup{e}}{4 \hbar}$.

\begin{table}
  \centering
  \caption{Darstellung der 16 Zustände des in der Spinausrichtung angeregten 8N4E-Systems im Hubbard-Modell aus Abschnitt \ref{sec:spin} mit der Vorschrift \eqref{eqn:8N4Espinvorschrift}.}
  \begin{minipage}[t]{0.35\linewidth}
    \begin{tabular}{S[table-format=2.0] S}
      \toprule
      {$i$} & {$\ket{i}$} \\
      \midrule
      1  &  $\ket{1;1}$ \\
      2  &  $\ket{1;2}$ \\
      3  &  $\ket{1;3}$ \\
      4  &  $\ket{1;4}$ \\
      5  &  $\ket{2;1}$ \\
      6  &  $\ket{2;2}$ \\
      7  &  $\ket{2;3}$ \\
      8  &  $\ket{2;4}$ \\
      \bottomrule
    \end{tabular}
  \end{minipage}
  \begin{minipage}[t]{0.35\linewidth}
    \begin{tabular}{S[table-format=2.0] S}
      \toprule
      {$i$} & {$\ket{i}$} \\
      \midrule
      9  &  $\ket{3;1}$ \\
      10 &  $\ket{3;2}$ \\
      11 &  $\ket{3;3}$ \\
      12 &  $\ket{3;4}$ \\
      13 &  $\ket{4;1}$ \\
      14 &  $\ket{4;2}$ \\
      15 &  $\ket{4;3}$ \\
      16 &  $\ket{4;4}$ \\
      \bottomrule
    \end{tabular}
  \end{minipage}
  \label{tab:spluszustände}
\end{table}


Die Matrixdarstellung des $S_+$-Operators aus Abschnitt \ref{sec:spin} zu den Basis-Zuständen aus Tabelle \ref{tab:spluszustände} lautet
\begin{align}
  \left( \mathcal{S}_+ \right)_{ij} =
  \begin{pmatrix*}[r]
    \mathcal{K} & \mathcal{L}
  \end{pmatrix*}_{ij}
  \label{eqn:splusmatrix}
\end{align}
mit

\begin{align*}
  \mathcal{K} &=
  \begingroup
    \setlength\arraycolsep{2.1pt}
    \begin{pmatrix*}[r]
      0 & 0 & 0 & 0 & 0 & 0 & 0 & 0 & 0 & 0 & 0 & 0 & 0 & 0 & 0 & 0 & 0 & 0 \\
      0 & 0 & 0 & 0 & 0 & 0 & 0 & 0 & 1 & 0 & 0 & 0 & 0 & -1 & 0 & 0 & 0 & 0 \\
      0 & 0 & 1 & 0 & 0 & 0 & 0 & 0 & 0 & 0 & 0 & 0 & -1 & 0 & 0 & 0 & 0 & 0 \\
      0 & 1 & 0 & 0 & 0 & 0 & -1 & 0 & 0 & 0 & 0 & 0 & 0 & 0 & 0 & 0 & 0 & 0 \\
      0 & 0 & 0 & 0 & 0 & 0 & 0 & 0 & 0 & 0 & 0 & 0 & 0 & 0 & 0 & 0 & 0 & \phantom{-}0 \\
      0 & 0 & 0 & 0 & 0 & 0 & 0 & 0 & 0 & 0 & 1 & 0 & 0 & 0 & 0 & -1 & \phantom{-}0 & 0 \\
      0 & 0 & 0 & 0 & 1 & 0 & 0 & 0 & 0 & 0 & 0 & 0 & 0 & 0 & \phantom{-}0 & 0 & 0 & 0 \\
      0 & 0 & 0 & 1 & 0 & 0 & 0 & 0 & 0 & 0 & 0 & \phantom{-}0 & 0 & 0 & 0 & 0 & 0 & 0 \\
      0 & 0 & 0 & 0 & 0 & 0 & 0 & 0 & 0 & 0 & \phantom{-}0 & 0 & 0 & 0 & 0 & 0 & 0 & 0 \\
      0 & 0 & 0 & 0 & 0 & 0 & 0 & 0 & 0 & \phantom{-}0 & 0 & 1 & 0 & 0 & 0 & 0 & 0 & 0 \\
      0 & 0 & 0 & 0 & 0 & 1 & 0 & 0 & \phantom{-}0 & 0 & 0 & 0 & 0 & 0 & 0 & 1 & 0 & 0 \\
      0 & 0 & 0 & 0 & 0 & 0 & 0 & \phantom{-}0 & 0 & 1 & 0 & 0 & 0 & 0 & 0 & 0 & 0 & 0 \\
      0 & 0 & 0 & 0 & 0 & \phantom{-}0 & 0 & 0 & 0 & 0 & 0 & 0 & 0 & 0 & 0 & 0 & 0 & 0 \\
      0 & 0 & 0 & 0 & \phantom{-}0 & 0 & 0 & 0 & 0 & 0 & 0 & 0 & 0 & 0 & 0 & 0 & 0 & 1 \\
      0 & 0 & \phantom{-}0 & \phantom{-}0 & 0 & 0 & 0 & 0 & 0 & 0 & 0 & 0 & 0 & 0 & 0 & 0 & 1 & 0 \\
      \phantom{-}0 & \phantom{-}0 & 0 & 0 & 0 & -1 & 0 & 0 & 0 & 0 & 1 & 0 & 0 & 0 & 0 & 0 & 0 & 0
    \end{pmatrix*}
  \endgroup\\
  \mathcal{L} &=
  \begingroup
    \setlength\arraycolsep{2.1pt}
    \begin{pmatrix*}[r]
      0 & 0 & 1 & 0 & 0 & 0 & 0 &-1 & 0 & 0 & 0 & 0 & 1 & 0 & 0 & 0 & 0 & 0 \\
      \phantom{-}0 & 0 & 0 & 0 & 0 & 0 & 0 & 0 & 0 & 0 & 0 & 0 & 0 & 0 & 0 & 0 & 0 & 0 \\
      0 & \phantom{-}0 & 0 & 0 & 0 & 0 & 0 & 0 & 0 & 0 & 0 & 0 & 0 & 0 & 0 & 0 & 0 & 0 \\
      0 & 0 & \phantom{-}0 & 0 & 0 & 0 & 0 & 0 & 0 & 0 & 0 & 0 & 0 & 0 & 0 & 0 & 0 & 0 \\
      0 & 0 & 0 & \phantom{-}0 & 1 & 0 & 0 & 0 & 0 &-1 & 0 & 0 & 0 & 0 & 0 & 0 & 0 & 0 \\
      0 & 0 & 0 & 0 & \phantom{-}0 & 0 & 0 & 0 & 0 & 0 & 0 & 0 &-1 & 0 & 0 & 0 & 0 & 0 \\
      0 & 0 & 0 & 0 & 0 & \phantom{-}0 &-1 & 0 & 0 & 0 & 0 & 0 & 0 & 0 & 0 & 0 & 0 & 0 \\
      1 & 0 & 0 & 0 & 0 & 0 & \phantom{-}0 & 0 & 0 & 0 & 0 & 0 & 0 & 0 & 0 & 0 & 0 & 0 \\
      0 & 0 & 0 & 0 & 0 & 1 & 0 & \phantom{-}0 & 0 & 0 & 0 & 0 & 0 & 0 & 0 &-1 & 0 & 0 \\
      0 & 0 & 0 & 0 & 0 & 0 & 0 & 0 & \phantom{-}0 & 0 & 0 & 0 & 0 &-1 & 0 & 0 & 0 & 0 \\
      0 & 0 & 0 & 0 & 0 & 0 & 0 &-1 & 0 & \phantom{-}0 & 0 & 0 & 0 & 0 & 0 & 0 & 0 & 0 \\
      0 &-1 & 0 & 0 & 0 & 0 & 0 & 0 & 0 & 0 & \phantom{-}0 & 0 & 0 & 0 & 0 & 0 & 0 & 0 \\
      0 & 0 & 0 & 0 & 0 & 0 & 0 & 0 & 0 & 0 & 0 & \phantom{-}1 & 0 & 0 & 0 & 0 &-1 & 0 \\
      0 & 0 & 0 & 0 & 0 & 0 & 0 & 0 & 0 & 0 & 0 & 0 & \phantom{-}0 & 0 &-1 & 0 & 0 & 0 \\
      0 & 0 & 0 & 0 & 0 & 0 & 0 & 0 &-1 & 0 & 0 & 0 & 0 & \phantom{-}0 & 0 & 0 & 0 & 0 \\
      0 & 0 &-1 & 0 & 0 & 0 & 0 & 0 & 0 & 0 & 0 & 0 & 0 & 0 & \phantom{-}0 & \phantom{-}0 & \phantom{-}0 & \phantom{-}0 \\
    \end{pmatrix*}.
  \endgroup
\end{align*}

Die Matrixdarstellung des Hamiltonoperators $H_\text{Diag}$ zur Beschreibung der Übernächst-Nachbar-Tight-Binding-Wechselwirkung aus Abschnitt \ref{sec:untersuchungife} zu den Basis-Zuständen aus Tabelle \ref{tab:8N4Ehubbzust} lautet
\setcounter{MaxMatrixCols}{18}
\begin{align}
  H_{ij} =
  \begin{pmatrix*}[l]
     \mathcal{X} & \mathcal{Y}^T \\
     \mathcal{Y} & \mathcal{Z}
  \end{pmatrix*}_{ij}
  \label{eqn:elmatrix}
\end{align}
mit
\begin{align*}
  \mathcal{X} & =
  \begingroup
    \setlength\arraycolsep{2.1pt}
    \begin{pmatrix*}[r]
      0 & 0 &-1 & 1 & 0 & 0 & 0 & 0 & 0 & 0 & 0 & 0 &-1 & 0 & 0 & 0 & 0 & 0 \\
      0 & 0 & 0 & 0 & 0 & 0 & 0 & 0 & 0 & 0 & 0 & 0 & 0 &-1 & 0 & 0 & 0 & 0 \\
      -1 &0 & 0 & 0 & 0 &-1 & 0 & 0 & 0 & 0 & 0 & 0 & 0 & 0 &-1 & 0 & 0 & 0 \\
      1 & 0 & 0 & 0 & 0 & 1 & 0 & 0 & 0 & 0 & 0 & 0 & 0 & 0 & 0 &-1 & 0 & 0 \\
      0 & 0 & 0 & 0 & 0 & 0 & 0 & 0 & 0 & 0 & 0 & 0 & 0 & 0 & 0 & 0 &-1 & 0 \\
      0 & 0 &-1 & 1 & 0 & 0 & 0 & 0 & 0 & 0 & 0 & 0 & 0 & 0 & 0 & 0 & 0 &-1 \\
      0 & 0 & 0 & 0 & 0 & 0 & 0 & 0 &-1 & 1 & 0 & 0 & 0 & 0 & 0 & 0 & 0 & 0 \\
      0 & 0 & 0 & 0 & 0 & 0 & 0 & 0 & 0 & 0 & 0 & 0 & 0 & 0 & 0 & 0 & 0 & 0 \\
      0 & 0 & 0 & 0 & 0 & 0 &-1 & 0 & 0 & 0 & \phantom{-}0 &-1 & 0 & 0 & 0 & 0 & 0 & 0 \\
      0 & 0 & 0 & 0 & 0 & 0 & 1 & 0 & 0 & \phantom{-}0 & 0 & 1 & 0 & 0 & 0 & 0 & 0 & 0 \\
      0 & 0 & 0 & 0 & 0 & 0 & 0 & \phantom{-}0 & 0 & 0 & 0 & 0 & 0 & 0 & 0 & 0 & 0 & 0 \\
      0 & 0 & 0 & 0 & 0 & 0 & 0 & 0 &-1 & 1 & 0 & 0 & 0 & 0 & 0 & 0 & 0 & 0 \\
      -1 &0 & 0 & 0 & 0 & 0 & 0 & 0 & 0 & 0 & 0 & 0 & 0 & 0 &-1 & 1 & 0 & 0 \\
      0 &-1 & 0 & 0 & 0 & 0 & 0 & 0 & 0 & 0 & 0 & 0 & 0 & 0 & 0 & 0 & 0 & 0 \\
      0 & 0 &-1 & 0 & 0 & 0 & 0 & 0 & 0 & 0 & 0 & 0 &-1 & 0 & 0 & 0 & 0 &-1 \\
      0 & 0 & 0 &-1 & 0 & 0 & 0 & 0 & 0 & 0 & 0 & 0 & 1 & 0 & 0 & 0 & 0 & 1 \\
      0 & 0 & 0 & 0 &-1 & 0 & 0 & 0 & 0 & 0 & 0 & 0 & 0 & 0 & 0 & 0 & 0 & 0 \\
      0 & 0 & 0 & 0 & 0 &-1 & 0 & 0 & 0 & 0 & 0 & 0 & 0 & 0 &-1 & 1 & 0 & 0 \\
    \end{pmatrix*}
  \endgroup \\
  \mathcal{Y} & =
  \begingroup
    \setlength\arraycolsep{2.1pt}
    \begin{pmatrix*}[r]
      \phantom{-}1 & 0 & 0 & 0 & 0 & 0 & 0 & 0 & 0 & 0 & 0 & 0 & 0 & 0 & 0 & 0 & 0 & 0 \\
      0 & \phantom{-}1 & 0 & 0 & 0 & 0 & 0 & 0 & 0 & 0 & 0 & 0 & 0 & 0 & 0 & 0 & 0 & 0 \\
      0 & 0 & \phantom{-}1 & 0 & 0 & 0 & 0 & 0 & 0 & 0 & 0 & 0 & 0 & 0 & 0 & 0 & 0 & 0 \\
      0 & 0 & 0 & \phantom{-}1 & 0 & 0 & 0 & 0 & 0 & 0 & 0 & 0 & 0 & 0 & 0 & 0 & 0 & 0 \\
      0 & 0 & 0 & 0 & \phantom{-}1 & 0 & 0 & 0 & 0 & 0 & 0 & 0 & 0 & 0 & 0 & 0 & 0 & 0 \\
      0 & 0 & 0 & 0 & 0 & \phantom{-}1 & 0 & 0 & 0 & 0 & 0 & 0 & 0 & 0 & 0 & 0 & 0 & 0 \\
      0 & 0 & 0 & 0 & 0 & 0 & \phantom{-}0 & 0 & 0 & 0 & 0 & 0 & 0 & 0 & 0 & 0 & 0 & 0 \\
      0 & 0 & 0 & 0 & 0 & 0 & 0 & \phantom{-}0 & 0 & 0 & 0 & 0 & 0 & 0 & 0 & 0 & 0 & 0 \\
      0 & 0 & 0 & 0 & 0 & 0 & 0 & 0 & \phantom{-}0 & 0 & 0 & 0 & 0 & 0 & 0 & 0 & 0 & 0 \\
      0 & 0 & 0 & 0 & 0 & 0 & 0 & 0 & 0 & \phantom{-}0 & 0 & 0 & 0 & 0 & 0 & 0 & 0 & 0 \\
      0 & 0 & 0 & 0 & 0 & 0 & 0 & 0 & 0 & 0 & \phantom{-}0 & 0 & 0 & 0 & 0 & 0 & 0 & 0 \\
      0 & 0 & 0 & 0 & 0 & 0 & 0 & 0 & 0 & 0 & 0 & \phantom{-}0 & 0 & 0 & 0 & 0 & 0 & 0 \\
      0 & 0 & 0 & 0 & 0 & 0 & 0 & 0 & 0 & 0 & 0 & 0 &-1 & 0 & 0 & 0 & 0 & 0 \\
      0 & 0 & 0 & 0 & 0 & 0 & 0 & 0 & 0 & 0 & 0 & 0 & 0 &-1 & 0 & 0 & 0 & 0 \\
      0 & 0 & 0 & 0 & 0 & 0 & 0 & 0 & 0 & 0 & 0 & 0 & 0 & 0 &-1 & 0 & 0 & 0 \\
      0 & 0 & 0 & 0 & 0 & 0 & 0 & 0 & 0 & 0 & 0 & 0 & 0 & 0 & 0 &-1 & 0 & 0 \\
      0 & 0 & 0 & 0 & 0 & 0 & 0 & 0 & 0 & 0 & 0 & 0 & 0 & 0 & 0 & 0 &-1 & 0 \\
      0 & 0 & 0 & 0 & 0 & 0 & 0 & 0 & 0 & 0 & 0 & 0 & 0 & 0 & 0 & 0 & 0 &-1 \\
    \end{pmatrix*}
  \endgroup
\end{align*}
\begin{align*}
  \mathcal{Z} &=
  \begingroup
    \setlength\arraycolsep{2.1pt}
    \begin{pmatrix*}[r]
      \phantom{-}0 & 0 &-1 & 1 & 0 & 0 & 0 & 0 & 0 & 0 & 0 & 0 & 1 & 0 & 0 & 0 & 0 & 0 \\
      0 & \phantom{-}0 & 0 & 0 & 0 & 0 & 0 & 0 & 0 & 0 & 0 & 0 & 0 & 1 & 0 & 0 & 0 & 0 \\
      -1& 0 & \phantom{-}0 & 0 & 0 &-1 & 0 & 0 & 0 & 0 & 0 & 0 & 0 & 0 & 1 & 0 & 0 & 0 \\
      1 & 0 & 0 & \phantom{-}0 & 0 & 1 & 0 & 0 & 0 & 0 & 0 & 0 & 0 & 0 & 0 & 1 & 0 & 0 \\
      0 & 0 & 0 & 0 & \phantom{-}0 & 0 & 0 & 0 & 0 & 0 & 0 & 0 & 0 & 0 & 0 & 0 & 1 & 0 \\
      0 & 0 &-1 & 1 & 0 & \phantom{-}0 & 0 & 0 & 0 & 0 & 0 & 0 & 0 & 0 & 0 & 0 & 0 & 1 \\
      0 & 0 & 0 & 0 & 0 & 0 & \phantom{-}0 & 0 &-1 & 1 & 0 & 0 & 0 & 0 & 0 & 0 & 0 & 0 \\
      0 & 0 & 0 & 0 & 0 & 0 & 0 & \phantom{-}0 & 0 & 0 & 0 & 0 & 0 & 0 & 0 & 0 & 0 & 0 \\
      0 & 0 & 0 & 0 & 0 & 0 &-1 & 0 & \phantom{-}0 & 0 & 0 &-1 & 0 & 0 & 0 & 0 & 0 & 0 \\
      0 & 0 & 0 & 0 & 0 & 0 & 1 & 0 & 0 & \phantom{-}0 & 0 & 1 & 0 & 0 & 0 & 0 & 0 & 0 \\
      0 & 0 & 0 & 0 & 0 & 0 & 0 & 0 & 0 & 0 & \phantom{-}0 & 0 & 0 & 0 & 0 & 0 & 0 & 0 \\
      0 & 0 & 0 & 0 & 0 & 0 & 0 & 0 &-1 & 1 & 0 & \phantom{-}0 & 0 & 0 & 0 & 0 & 0 & 0 \\
      1 & 0 & 0 & 0 & 0 & 0 & 0 & 0 & 0 & 0 & 0 & 0 & \phantom{-}0 & 0 &-1 & 1 & 0 & 0 \\
      0 & 1 & 0 & 0 & 0 & 0 & 0 & 0 & 0 & 0 & 0 & 0 & 0 & \phantom{-}0 & 0 & 0 & 0 & 0 \\
      0 & 0 & 1 & 0 & 0 & 0 & 0 & 0 & 0 & 0 & 0 & 0 &-1 & 0 & \phantom{-}0 & 0 & 0 &-1 \\
      0 & 0 & 0 & 1 & 0 & 0 & 0 & 0 & 0 & 0 & 0 & 0 & 1 & 0 & 0 & \phantom{-}0 & 0 & 1 \\
      0 & 0 & 0 & 0 & 1 & 0 & 0 & 0 & 0 & 0 & 0 & 0 & 0 & 0 & 0 & 0 & \phantom{-}0 & 0 \\
      0 & 0 & 0 & 0 & 0 & 1 & 0 & 0 & 0 & 0 & 0 & 0 & 0 & 0 &-1 & 1 & 0 & \phantom{-}0 \\
    \end{pmatrix*}
  \endgroup
\end{align*}
in Einheiten von $\lambda = 0.2 \, J$.

\newpage

\section{Gitterplatz-Vektoren und Anfangsbedingungen für das zeitabhängige System}

Die Ortsvektoren der Gitterplätze aus Abschnitt \ref{sec:untersuchungife}, welche in Gleichung \eqref{eqn:qPhi} benötigt werden, lauten
\begin{align}
  \vec{r}_1 &= \frac12
  \begin{pmatrix*}[r]
     -1 \\
     -1
  \end{pmatrix*} &
  \vec{r}_2 &= \frac12
  \begin{pmatrix*}[r]
     1 \\
     -1
  \end{pmatrix*} &
  \vec{r}_3 &= \frac12
  \begin{pmatrix*}[r]
     \phantom{-}1 \\
     1
  \end{pmatrix*} &
  \vec{r}_4 &= \frac12
  \begin{pmatrix*}[r]
     -1 \\
     1
  \end{pmatrix*}
  \label{eqn:gitterplatzvektor}
\end{align}
in Einheiten von $a$.

\begin{table}
  \centering
  \caption{Koeffizienten $\alpha_i$ der Linearkombination der Basis-Zustände für den Eigenzustand $\Psi_0$ des 8N4E-Systems
  jeweils mit $U/J = 4$ und $U/J = 8$. Die Basis-Zustände $\ket{i}$ sind in Tabelle \ref{tab:8N4Ehubbzust} aufgelistet.}
  \begin{minipage}[t]{0.35\linewidth}
    \begin{tabular}{S[table-format=2.0] S[table-format=2.3] S[table-format=2.3]}
      \toprule
      {$i$} & {$\alpha_i(4)$} & {$\alpha_i(8)$} \\
      \midrule
      1  & -0.051 & -0.021 \\
      2  & -0.129 & -0.089 \\
      3  &      0 &      0 \\
      4  &      0 &      0 \\
      5  &  0.129 &  0.089 \\
      6  &  0.245 &  0.270 \\
      7  & -0.129 & -0.089 \\
      8  &      0 &      0 \\
      9  &  0.129 &  0.089 \\
      10 &  0.129 &  0.089 \\
      11 &  0.491 &  0.539 \\
      12 &  0.129 &  0.089 \\
      13 &      0 &      0 \\
      14 &  0.129 &  0.089 \\
      15 &  0.051 &  0.021 \\
      16 &  0.245 &  0.270 \\
      17 &  0.129 &  0.089 \\
      18 &      0 &      0 \\
      \bottomrule
    \end{tabular}
  \end{minipage}
  \begin{minipage}[t]{0.35\linewidth}
    \begin{tabular}{S[table-format=2.0] S[table-format=2.3] S[table-format=2.3]}
      \toprule
      {$i$} & {$\alpha_i(4)$} & {$\alpha_i(8)$} \\
      \midrule
      19 &      0 &      0 \\
      20 &  0.129 &  0.089 \\
      21 &  0.245 &  0.270 \\
      22 &  0.051 &  0.021 \\
      23 &  0.129 &  0.089 \\
      24 &      0 &      0 \\
      25 &  0.129 &  0.089 \\
      26 &  0.491 &  0.539 \\
      27 &  0.129 &  0.089 \\
      28 &  0.129 &  0.089 \\
      29 &      0 &      0 \\
      30 & -0.129 & -0.089 \\
      31 &  0.245 &  0.270 \\
      32 &  0.129 &  0.089 \\
      33 &      0 &      0 \\
      34 &      0 &      0 \\
      35 & -0.129 & -0.089 \\
      36 & -0.051 & -0.021 \\
      \bottomrule
    \end{tabular}
  \end{minipage}
  \label{tab:Upsi0}
\end{table}

\newpage

\section{Eigenenergien des modellierten Mott-Hubbard-Isolators}

\begin{table}[H]
  \centering
  \caption{Datenpunkte der sechs niedrigsten Energieeigenwerte der Matrix \eqref{eqn:8N4Ehubbmatrix} des 8N4E-Systems im Hubbard-Modell aus Abschnitt \ref{sec:untersuchunghubb} in Abhängigkeit vom Potential $U$.}
  \begin{tabular}{S[table-format=1.0] S[table-format=2.2] S[table-format=2.2] S[table-format=2.2] S[table-format=2.2] S[table-format=2.2] S[table-format=2.2]}
    \toprule
    {$U/J$} & {$E_0/J$} & {$E_1/J$} & {$E_2/J$} & {$E_3/J$} & {$E_4/J$} & {$E_5/J$} \\
    \midrule
    1  & -3.34 & -3.29 & -2.86 & -2.79 & -1.56 & -1.56 \\
    2  & -2.83 & -2.69 & -2.00 & -1.63 & -1.24 & -1.24 \\
    3  & -2.42 & -2.19 & -1.42 & -1.00 & -1.00 & -0.51 \\
    4  & -2.10 & -1.81 & -1.07 & -0.83 & -0.83 & 0     \\
    5  & -1.84 & -1.51 & -0.84 & -0.70 & -0.70 & 0     \\
    6  & -1.63 & -1.29 & -0.70 & -0.61 & -0.61 & 0     \\
    7  & -1.46 & -1.12 & -0.59 & -0.53 & -0.53 & 0     \\
    8  & -1.32 & -0.99 & -0.51 & -0.47 & -0.47 & 0     \\
    9  & -1.20 & -0.88 & -0.45 & -0.42 & -0.42 & 0     \\
    10 & -1.10 & -0.80 & -0.41 & -0.39 & -0.39 & 0     \\
    \bottomrule
  \end{tabular}
  \label{tab:eigenwerte}
\end{table}

\begin{figure}[H]
  \centering
  \includegraphics[height=7.5cm]{build/Hubb_eplot2.pdf}
  \caption{Die ersten sechs Eigenenergiewerte des 8N4E-Systems für Abschnitt \ref{sec:spin} in Einheiten von $J$ und in Abhängigkeit von $U/J$. Sie sind zu den Eigenvektoren $\Psi_q$ geordnet, welche wiederum so geordnet sind,
  dass sich die Spins $S_q$ mit $U/J$ nicht ändern. Die Energien $E_4$ ist zweifach entartet.}
  \label{fig:eplot2}
\end{figure}

\begin{figure}[H]
  \centering
  \includegraphics[height=7.5cm]{build/Hubb_ediff_plot.pdf}
  \caption{Eigenenergiedifferenz $\Delta E_{0 \to 2} = E_2 - E_0$ in Einheiten von $J$ und in Abhängigkeit von $U/J$. Die Eigenzustände zu $E_0$ und $E_2$ haben jeweils den Gesamtspin $S=0$.}
  \label{fig:ediffplot}
\end{figure}

\section{Lineare Regression der Grundzustandsenergie}
\label{sec:linreg}

Der Logarithmus der Grundzustandsenergie wird insgesamt für 10001 Potentialwerte mit äquidistantem Abstand von $U/J = \num{40000}$ bis $U/J = \num{50000}$ linear approximiert.
Zur Geradengleichung
\begin{align}
  \ln{\frac{E_\text{0,hubb}}{J}} = m \ln{\frac{U}{J}} + n
  \label{eqn:loge0hubbgerade}
\end{align}
werden die Regressionsparameter
\begin{align}
  m & = -1 & n & = 2.484906
  \label{eqn:regressionsparameter}
\end{align}
berechnet. Die Fehler der Parameter werden nicht explizit angegeben, da sie in der Größenordnung $10^{-9}$ liegen.
Es wird ein Koeffizientenvergleich zwischen Gleichung \eqref{eqn:loge0hubb} und \eqref{eqn:loge0hubbgerade} durchgeführt.
Die Steigungen stimmen überein und für den Koeffizienten ergibt sich
\begin{align}
   \eta_\text{Hubb} = 3 - \num{2e-06}.
\end{align}
In Abbildung \ref{fig:e0plot} ist die Grundzustandsenergie des 8N4E-Systems im Hubbard-Modell in Einheiten von $J$ gegen das Ergebnis der Störungsrechnung zweiter Ordnung
\begin{align*}
  E_0/J = - 3 t/J = - 3 \frac{4J}{U}
\end{align*}
aufgetragen. In dieser Graphik ist erneut erkennbar, dass sich beide Kurven für $U/J \to \infty$ asymptotisch annähern.

\begin{figure}[H]
  \centering
  \includegraphics[height=7.5cm]{build/heis_hubb_vgl_plot.pdf}
  \caption{Grundzustandsenergie des 8N4E-Systems, im Hubbard-Modell exakt und in zweiter Ordnung Störungstheorie genähert, $E_{0\text{,Korr}} = -3t$, in Einheiten von $J$.}
  \label{fig:e0plot}
\end{figure}
